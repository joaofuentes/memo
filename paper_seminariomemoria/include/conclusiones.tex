\section{Conclusiones}

Día a día la tecnología avanza, y la Web con ella ha ido progresando año tras año. Las redes sociales ya no son ninguna, 
las páginas Web hace tiempo que dejaron de ser simples vitrinas de información estática, el dinamismo en la 
Web es total y las aplicaciones de escritorio han comenzado el éxodo hacia Internet en forma de herramientas 
colaborativas. 

La evolución es la suma de la imaginación e investigación. Mientras que la imaginación permite adelantar técnologías, soñar con cosas y 
elementos impensados para la época, la investigación sumada al desarrollo, permite realizar todos estos sueños, y a su vez, soñar con
cosas nuevas cosas; completando así el ciclo de la evolución.

De la investigación realizada se obtuvieron las siguientes conclusiones:
\begin{enumerate}
 \item La evolución es algo inherente al ser humano, la cual surge debido a la complejización de procesos.
 \item Dicha evolución se puede dar, gracias al desarrollo, creación e investigación de las nuevas técnologías.
 \item Las nuevas técnologías aparecen debido a la necesidad de solucionar problemas y/o resolver nuevas necesidades,
	no siendo éstas las únicas razones.
 \item El campo de las aplicaciones web no es la excepción, dicha evolución es visible al ver los cambios que van desde la
	llamada web1.0 hasta las técnologías de hoy. 
 \item Dicha evolución está gatillada por nuevas necesidades que aparecen gracias
	a la evlución de otros campos, tales como la física, electrónica (creación de nuevo HW) y cambios en la sociedad. 
 \item El campo de las aplicaciones web se encuentra en plena revolución, en lo que algunos expertos llaman la transición de la
	web 3.0, debido a las nuevas técnologías que han ido apareciendo (para satisfacer nuevas necesidades)
 \item A pesar de las opiniones y declaraciones de diversos expertos, aún no se tiene pleno consenso sobre la situación actual del
	desarrollo web.
 \item Es de suma importancia investigar y evaluar éstas técnologías\footnote{A nivel de memoria de pregrado se realizará esta 
	investigación a algunas de ellas}. 
\end{enumerate}



%\textbf{Actividad Realizado: Tiempo utilizado}
%
%Busqueda y selección de información: \\
%
%Lectura y análisis en detalle de la información recolectada:  \footnote{A medida que se extraía información de 
%las diversas fuentes, se redactaban resumenes con el objetivo de comprender mejor su contenido}\\
%
%Desarrollo del documento: \footnote{Con los resumenes listos, surgía la necesidad de complementar
%la información con otras fuentes.} \\
%
%Revisión Ortográfica y redacción:\\
%
%Total:\\
