\section{Arquitectura}

%explicar en parrafo pequeño que es una arquitectura de software
%\textit{agregar parrafo introductorio}\\
Una arquitectura de software corresponde a un patrón de referencia que brinda
un marco necesario para guiar la construcción de software y establece la estrucutura de 
funcionamiento  e interacción entre sus diversas partes [20]. Este concepto 
"se refiere a la estructuración del sistema que, idealamente, se crea en las etapas 
tempranas de desarrollo"[21].\\ 

Al crear un software (independientemente de la metodología que se utilice), es necesario
cumplir una serie de pasos  que preceden a su construcción:
\begin{itemize}
 \item Requerimientos: Se enfoca a la captura y priorización de necesidades a satisfacer,
  ya sean de calidad, rendimiento y/o reestrictivas. Estos requerimientos son preponderantes
  e influencia la desición acerca de que arquitectura a utilizar.
 \item Diseño: Corresponde a la fase central en relación con la arquitectura. Debe satisfacer todos
  los requerimientos y no solo utilizar tecnologías de moda.
 \item Documentación: La documentación un factor crucial para comunicar un diseño de forma exitosa.
  Generalemente se utiliza la representación de varias de sus estructuras mediante el uso vistas. 
  Una vista generalmente contiene un diagrama, además de información adicional, que apoya en la 
  comprensión de dicho diagrama.
 \item Evaluación: Es de suma importancia evaluar el diseño una vez que este ha sido documentado 
  con el fin de identificar posibles problemas y riesgos. Evaluar (y validar) el diseño antes
  de codificar, disminuye el costo de corrección de errores.
\end{itemize}


Los grandes objetivos de una arquitectura de software son :
\begin{enumerate}
 \item Servir como guía durante el proceso de desarrollo. 
 \item Definir y satisfacer los atributos de calidad. 
\end{enumerate}
  

%explicar pq es necesario definir una arquitectura
Como dice Danny Thorpe:
\begin{center}
 \textit{"Programar sin una arquitectura en mente, es como explorar una gruta sólo con una
 linterna: no sabes dónde estás, dónde has estado ni hacia dónde vas"} [18]
\end{center}
La importancia de contar con una buena arquitectura que se adecue a situaciones reales, donde 
las diversas herramientas serán probadas. Es por ello que la arquitectura debe ser la respuesta 
ante un problema, no una imposición. Por otra parte, el desarrollo de software dejó de ser, hace mucho
tiempo, el trabajo de una o dos personas, pasando a ser equipos; por tanto es necesario facilitar la
comunicación entre sus integrantes.\\

A la hora de diseñar un software, hay que tener en cuenta que la arquitectura a utilizar se 
describe utilizando varios tipos de modelos[22]:
\begin{itemize}
 \item Estructurales: Se centran en la estructura coherente del sistema completo, en lugar de centrarse 
  en su composición. Representan todo como una colección organizada de componentes. 
 \item Frameworks: Identifican patrones de diseño repetibles, los cuales se pueden encontrar en aplicaciones
  similares.
 \item Dinámicos: Se centran en los aspectos del comportamiento dinámico de la arquitectura,  indicando
  como la estructura o la configuración del sistema puede cambiar en función de eventos externos.
 \item De Procesos: Se enfocan en el diseño de los procesos del negocio que el sistema debe soportar.
\end{itemize}

Dependiendo de la situación, se debe tomar la desición de que tipo de arquitectura utilizar.\\



Algunos elementos de una arquitectura son:
\begin{itemize}
 \item Cliente y Servidor
 \item Base de Datos
 \item Componentes
 \item Sistemas de Nivel Jerárquico
\end{itemize}

De entre las diversas arquitecturas, algunas de las más relevantes son:
\begin{itemize}
 \item Cliente-Servidor
 \item Orientado a Servicios
 \item Modelo Vista Controlador
 \item Modelo de N-Capas
\end{itemize}
%\textit{Pregntar al profe si es necesario explicar cada uno} basta con explicar y adjuntar diagrama




Considerando el hecho de que uno de los objetivos de esta memoria es, realizar pruebas y obtener
resultados concretos acerca de diversas herramientas de desarrollo web, es necesario enmarcar
dichas pruebas bajo un estándar que asegure una base pareja para todas ellas.\\

Es por dicha razón que se utilizará una arquitectura basada en 3 capas, de tipo \textbf{estructural} y 
\textbf{framework}.\\

\subsection{Arquitectura de 3 capas}
Una arquitectura bastante popular y de carácter relativamente simple, es la de 3 capas, cuyo
objetivo primordial es la separación de la aplicación en [23]:

\textit{Crear y agregar una imagen (con su respectiva explicación)}\\

\begin{enumerate}
 \item Una capa de presentación: Corresponde a la capa que ve el usuario, es decir, presenta 
  el sistema al usuario. Se responsabiliza de que se le comunique información al usuario por 
  parte del sistema y viceversa. La tarea de capturar información sobre el usuario se realiza,
  por lo general a través de formularios. Esta capa se comunica exclusiamente con la capa lógica.
  
 \item Una capa lógica/de negocios: Es la encargada de recibir, procesar y responder las peticiones
  que recibe, tanto del usuario a través de la capa de presentación, como las de la Base de Datos mediante
  la capa de datos. En otras palabras es una capa intermedia.
 
 \item Una capa de datos: Es la encargada acceder a los datos y está formada por un (o más) gestor de 
  base de datos. Dicho gestor recibe las peticiones de almacenamiento, recuperación, actualización y/o 
  eliminación de datos; desde la capa lógica. 
\end{enumerate}

Basicamente el objetivo principal es separar las distintas lógicas de la aplicación en niveles y que
éstos posean estructuras bien planteadas.\\

Para el caso de esta memoria, las pruebas se centrarán en la capa lógica y de presentación. Dejando
la capa de datos como una constante.
%Uso de arquitectura 3-tier
  %explicar de que se trata y pq se usa
  %hacer el diagrama 

%mapear las capas a la aplicaciòn



\newpage