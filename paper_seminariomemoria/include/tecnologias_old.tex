\section{Tecnologías utilizadas en el Desarrollo de aplicaciones Web}
\subsection{Un poco de historia}
Antiguamente, la creación de un sitió web se limitaba sólo  a escribir cada página directamente con mediante
código HTML. Esta tarea es factible solamente en sitios cuyo contenido es limitado y sus actualizaciones son casi nulas,
características propias de la \textit{Web 1.0} es decir, las típicas  que ostentaban los sitios durante 
la primera mitad de los años 90. 

Posteriormente aparecen los lenguajes de desarrollo Web intentan facilitar las tareas de los creadores de aplicaciones, 
de manera que se automaticen los procesos, y permitan entrar al juego a los usuarios, pasivos hasta ese momento, es decir
se crea la web 2.0

Por lo tanto, mientras que con HTML sólo es posible crear sitios Web estáticos, utilizando lenguajes 
de desarrollo Web es posible crear sitios Web dinámicos. Se conoce con el nombre de sitio Web dinámico a aquel
cuyo contenido se genera a partir de lo que un usuario introduce en un web o formulario. 


\subsection{Web 1.0}
\subsubsection{Características}
La Web 1.0 o "web estática"\footnote{El termino estático se refiere a su falta de interacción con el usuario. Es decir que 
el mismo sólo podía comunicarse con el desarrollador de la página a través del correo electrónico.} (1991-2003) es la forma más básica 
que existe, con navegadores de sólo texto bastante rápidos. La aparición de HTML hizo que las páginas web fuesen más agradables 
a la vista. Paralelamente aparecen los primeros navegadores visuales tales como Internet Explorer y Netscape [3]. Su principal
característica es que es de sólo lectura, es decir que para el usuario no es posible interactuar con el contenido de la página 
estando totalmente limitado a lo que el Webmaster sube a ésta.

\subsubsection{Tecnologías de desarrollo}
La tecnología predominante en la web 1.0, es el código HTML.

HTML corresponde al acrónimo de \textit{HyperText Markup Language} y se trata de un conjunto de etiquetas que sirven para 
definir el texto y otros elementos que compondrán una página web. Básicamente este lenguaje indica a los navegadores cómo 
deben mostrar el contenido de una página web.

HTML se creó con el objetivo de divulgar información, principalmente texto y posteriormente texto con imágenes. 
Creado en 1986 por el físico nuclear Tim Berners-Lee [9], no se pensó que llegaría  a ser utilizado para crear sitios de 
consulta con carácter multimedia. Sin embargo, pese a esta deficiente planificación, se han ido incorporando modificaciones 
con el tiempo, estos son los estándares del HTML\footnote{El estándar actual corresponde a HTML5} [8].

Desde el punto de vista del webmaster, la tarea de mantención de la página es relativamente simple, considerando la base de 
la web 1.0. Sin embargo, se convierte en una tarea titánica en aquellos sitios con muchos contenidos y que incorporan 
frecuentemente novedades. Por ejemplo, si se quieren realizar en HTML cambios sobre algún elemento común a todas las páginas 
del sitio, se deben aplicar en todas las páginas, una por una, con lo que se  convierte en un trabajo muy tedioso. Por lo cual
nace la necesidad de integrar al usuario a la faceta de creación y mantención de contenidos.

\subsection{Web 2.0}
\subsubsection{Características}

El término Web 2.0 está asociado a aplicaciones web que están desarrolladas para compartir información, 
pues su diseño está centrado en el usuario. Un sitio Web 2.0 está pensado para que los usuarios puedan interactuar 
y colaborar entre sí [4], tomando el rol de creadores de contenido generado por ellos mismos en una comunidad virtual,
lo cual es diametralmente opuesto al concepto de pasividad del usuario, algo predominante en la web 1.0.

Por lo tanto, la Web 2.0 es una evolución del viejo concepto de cómo se usa la web, de manera
unidireccional, como consumidores pasivos. El término, acuñado por Tim O'Reilly\footnote{En [6], realiza interesantes 
comparaciones entre los elementos de la web 1.0 y la web 2.0} en una conferencia del renacimiento y la evolución de la web, 
designa una nueva forma de servicios web basados en la participación de los usuarios, quienes conforman el motor básico del 
sistema de información.

La Web 2.0 se caracteriza por premiar la creatividad de los usuarios, fomentar su participación y transmisión del conocimiento 
entre pares potenciando el sentido de comunidad. En esta nueva época la Web ya no es meramente informativa, sino que 
el usuario toma rol consumidor, intercambiador y creador de contenidos. Los sitios web se convierten en fuentes de contenido y expresión
para los usuarios. Es aquí donde de fundan los cimientos de la Web 2.0: Arquitectura de participación, redes sociales y 
sabiduría de las multitudes, tan sólo por nombrar algunas.

Debido al aumento en la participación de los usuarios, nacen premisas como: “todos tienen algo que decir y todos pueden hacerlo” 
(Orihuela, 2006). El volumen de datos generados es tal, que se necesitan sistemas de filtrado, clasificación y 
organización de la información; sistemas que además, deben estar basados en la arquitectura de la participación y la inteligencia 
colectiva [5].

%El hecho de que la Web 2.0 es cualitativamente diferente de las tecnologías web anteriores ha sido cuestionado por el creador de 
%la World Wide Web Tim Berners-Lee, quien calificó al término como "tan sólo una jerga"- precisamente porque tenía la intención de que 
%la Web incorporase estos valores en el primer lugar. 

\subsubsection{Tecnologías de desarrollo}
Ante la necesidad de empoderar a los usuarios, fueron surgiendo varios lenguajes de programación y tecnologías orientadas
al desarrollo web necesarios para lograr la creación de sitios dinámicos.

Un sitio web dinámico se puede generar a través de secuencias de comandos en un servidor web; cuando el cliente web recibe la 
respuesta, la trata como una pagina HTML y la despliega. Un ejemplo de esto es cuando un usuario rellena los campos de un formulario y
realiza el envío de la información, al momento de llegar al servidor dicha información se entrega a un programa o secuencia de comandos
para que sea procesada, que por lo general corresponde a una interacción con una base de datos y la generación de una página HTML
con información personalizada, la cual es reenviada al cliente.

Por lo general, los sitios web dinámicos, estan compuestos por la combinación:
\begin{itemize}
 \item Plataforma del servidor web: Apache, Tomcat, entre otros.
 \item Gestor de base de datos, que por lo general es de carácter relacional: Oracle, PostgreSQL, Microsoft SQL Server, MySQL, entre otros.
 \item Lenguajes y tecnologías de programación web: Perl, PHP, JSP, CGI, JQuery, entre otros.
\end{itemize}

Dependiendo de las necesidades, se define la combinación adecuada. En este documento se revisarán algunas de los Lenguajes y
tecnologías para el desarrollo web, relegando tanto la plataforma del servidor como el gestor de base de datos, pues se escapan 
del alcance de la investigación.

Algunas de las tecnologías de la llamada web 2.0 son:

\begin{itemize}
 \item CGI: Una forma tradicional a la hora de manejar formularios y páginas web interactivas, corresponde a un sistema denominado
	    CGI \footnote{Common Gateway Interface por sus siglas en inglés, o Interfaz de Puerta de Enlace Común.}. En las aplicaciones
	    CGI, el servidor web pasa las solicitudes del cliente a un programa externo.  La salida de dicho programa es enviada al cliente 
	    en lugar del archivo HTML tradicional. El cliente se encarga de interpretar esta salida.

 \item PHP: Otra forma de generar el contenido dinámico, corresponde a que sea el servidor quien ejecute las secuencias de comandos para
	    generar la página HTML. PHP \footnote{Hypertext Pre-processor por sus siglas en inglés.}  tiene la ventaja de ser gratuito y 
	    versátil, pues es soportado por la mayoría de los sistemas operativos y servidores; además de contar con múltiples herramientas
	    de desarrollo como frameworks\footnote{Framework es un concepto sumamente genérico, se refiere a “ambiente de trabajo, 
	    y ejecución”.En general los framework son soluciones completas que contemplan herramientas de apoyo a la construcción 
	    (ambiente de trabajo o desarrollo) y motores de ejecución (ambiente de ejecución).} donde destacan PHPCake o Symfony. 
	    Para utilizar PHP, el servidor Web debe entenderlo. Por lo general, las páginas Web que contienen comandos PHP utilizan la
	    extensión “.php” en lugar de “.html”. De todos modos, el cliente nunca ve el código PHP, sino los resultados que produce
	    en código HTML.
    

 \item JavaScript: A pesar de la gran potencia de las tecnologías anteriores, ninguna de ellas puede responder, por ejemplo,
		   a los movimientos del ratón o interactuar de manera directa con los usuarios.
		   Para lograr esto, es necesario tener secuencias de comandos embebidas en las páginas HTML, pero que a diferencia
		   de, por ejemplo PHP,  se ejecuten en la máquina cliente y no en el servidor.
		   JavaScript es un lenguaje de scripts interpretado que se integra directamente en páginas HTML (a veces por 
		   modularidad se separa en ficheros con extensión “.js”) y es interpretado, en su totalidad, por el cliente Web 
		   en tiempo de ejecución, sirviendo así para todos los sistemas operativos.
 
 %\item Blogs y Redes sociales:
 \end{itemize}

\subsection{Evolución}
\subsubsection{Características}
Actualmente se está viviendo otra revo-lución, términos como web semántica y web 3.0 son cada vez más comunes. ¿Pero a qué se refieren?

A fines de la década de 1990, comenzó a idearse un nuevo cambio en la Web. Era un cambio a la vez más complejo y más profundo que el que 
ha representado la Web 2.0. Se trataba del proyecto de la Web Semántica.

A grandes rasgos, el objetivo fundacional de la Web Semántica consistió en desarrollar una serie de tecnologías que permitieran a las
computadoras, a través del uso de agentes de usuarios parecidos a los navegadores actuales, no solo “entender” el contenido de las 
páginas web, sino además efectuar razonamientos sobre el mismo. La idea era conseguir que el enorme potencial de conocimiento
encerrado en documentos como las páginas web pudiera ser interpretado por las computadoras de una forma parecida a como lo haría 
un ser humano.[14]

Utilizar números para marcar generaciones sucesivas de la Web parece una buena idea cuando se comprueba el éxito que tuvo la denominación 
2.0. ¿Cuáles serían los rasgos de esta futura Web? Aquí se entra en un terreno difícil, puesto que no se trata de algo existente en la
actualidad, sino de una especulación acerca de cómo se va encaminado la Web ahora y en el futuro cercano. Una forma de solucionar el 
problema es lo que hacen algunos analistas y que consiste en identificar Web 3.0 con Web Semántica. Otros analistas los ven de forma 
separada, donde la web 3.0 tiene las siguientes características:
\begin{itemize}
 \item Computación en la nube y Bases de datos no relacionales
 \item Agentes de usuario (como en la Web Semántica)
 \item Anchura de banda
 \item Mayor acceso a internet
\end{itemize}

Si bien los dos últimos puntos son de carácter técnico, sin duda están teniendo repercusiones sociales. A mayor ancho de banda se facilita
la ejecución de aplicaciones multimedia, mientras que un mayor acceso a internet no implica sólo que hay mayor cantidad de conexiones,
sino que es posible conectarse desde toda clase de aparatos electrónicos, como celulares, tablets e incluso vehículos. Esto a su
vez incentiva la creación e investigación de nuevas tecnologías a la hora de desarrollar aplicaciones web, ya sea para estandarizar
aspectos entre dispositivos, plataformas, sistemas operativos; o bien para personalizar cada aplicación con el fin de sacarle el máximo
provecho en un dispositivo en particular.

De todos modos y en definitiva, de eso trata la Web 3.0, de páginas capaces de comunicarse con otras páginas mediante procesamiento 
de lenguaje natural y, es justo aquí cuando cobra sentido el nexo entre la Web Semántica y la Web 3.0. Ésta es la principal interpretación
que se hace de éste término. 

\subsubsection{Tecnologías de desarrollo}

Algunas de las tecnologías de reciente aparición (y no tan recientes), de carácter libre y que se están aplicando en el 
área del desarrollo web, son:

\begin{itemize}
 \item QOOXDOO \footnote{http://qooxdoo.org/}: Qooxdoo es un framework universal de JavaScript que permite crear aplicaciones para 
  una amplia gama de plataformas. Qooxdoo aprovecha las tecnologías web más modernas, como HTML5y CSS3. Es de código abierto,
  está totalmente basado en clases y trata de aprovechar las características de orientación a objetos de JavaScript. 
  Se basa completamente en los espacios de nombres y no se extiende tipos nativos de JavaScript para permitir una fácil integración 
  con otras bibliotecas y código de usuario existente.[7]
  Una aplicación típica de qooxdoo se crea mediante el aprovechamiento de las herramientas de desarrollo integrado y el modelo de 
  programación del lado del cliente basada en orientación a objetos de JavaScript.
  Algunas de sus características son:
   \begin{enumerate}
    \item Qooxdoo soporta una amplia gama de entornos de JavaScript, tales como navegadores convencionales
      \footnote{Internet Explorer, Firefox, Opera, Safari, Chrome} y móviles \footnote{iOS, Android}
    \item No necesita plugins (ActiveX, Flash, Silverlight)
    \item Mantiene objetos nativos de JavaScript, con el fin de permitir una fácil integración con bibliotecas y código personalizado. 
    \item Al estar bajo el paradigma de la orientación a objetos, está basado en clases (en su totalidad). Además soporta
      clases abstractas.
    \item Cuenta con soporte completo para programación basada en eventos
    \item El desarrollo de aplicaciones qooxdoo es totalmente compatible con todas las plataformas, como Windows, todos los 
      sistemas Unix (Linux), Mac OS X.
    \item Cuenta con muchas aplicaciones de muestra y ejemplos.
   \end{enumerate}
 \item Node.js\footnote{http://nodejs.org/}: Es un entorno de programación basado en el lenguaje de programación 
  Javascript, con I/O de datos y una arquitectura orientada a eventos. Fue creado con el enfoque de ser
  útil en la creación de programas de red altamente escalables, como por ejemplo, servidores web.
  Las primeras encarnaciones de JavaScript vivían en los browsers, es decir en el frontend. Sin embargo, lo anterior es solo un contexto,
  pues JavaScript es un lenguaje "completo"; es decir, se puede  usar en muchos contextos y alcanzar con éste, todo lo que se puede
  alcanzar con cualquier otro lenguaje "completo".
  Por ello Node.js realmente es sólo otro contexto: permite correr código JavaScript en el backend, fuera del browser.
  Para ejecutar el código JavaScript que se pretende correr en el backend, debe ser interpretado y ejecutado. Node.js se encarga
  de esta tarea haciendo uso de la Maquina Virtual V8 de Google; que por lo demás es el mismo entorno de ejecución para JavaScript 
  que Google Chrome utiliza.
  Además, Node.js viene con muchos módulos útiles, de manera que no hay que escribir todo de cero.
 

 \item V8 Engine \footnote{http://code.google.com/p/v8/}: Es un motor de JS desarrollado por Google. La 
 ejecución de programas JS se realiza compilando el código, aumentando el desempeño respecto a Java ejecutado en lenguaje 
 interpretado Bytecode.
 Algunas de sus características son:
 \begin{enumerate}
  \item Está escrito en C++ y es usado en Google Chrome.
  \item Está integrado en el navegador de internet del sistema operativo Android, al menos desde su versión Froyo.+
  \item Corre en Windows (desde la versión XP), Mac OS X 10.5 (Leopard) y Linux en procesadores IA-32 y ARM.
  \item V8 puede funcionar de manera individual (standalone) o incorporada a cualquier aplicación C++.
 \end{enumerate}

 \item MongoDB \footnote{http://www.mongodb.org/}: MongoDB es un sistema de base de datos multiplataforma orientado a documentos, 
 de esquema libre. Al estar escrito en C++ le confiere cierta cercanía a los recursos de hardware de la máquina, de 
 modo que es bastante rápido a la hora de ejecutar sus tareas. 
 Al ser NoSQL, hay que olvidarse de las tablas y las relaciones entre ellas.
 En MongoDB, cada registro o conjunto de datos se denomina documento. Los documentos se pueden agrupar en colecciones, las cuales se 
 podría decir que son el equivalente a las tablas en una base de datos relacional (sólo que las colecciones pueden almacenar 
 documentos con muy diferentes formatos, en lugar de estar sometidos a un esquema fijo). Se pueden crear índices para algunos 
 atributos de los documentos, de modo que MongoDB mantendrá una estructura interna eficiente para el acceso a la información por 
 los contenidos de estos atributos. [13] 
 Se abandona el enfoque relacional por bases de datos mas orientadas a objetos y de esta manera es como se procesa la información.
 
 \item Kendo \footnote{http://www.kendoui.com/}: Bajo el lema “El arte del desarrollo web” Kendo UI ofrece un completo abanico 
 de posibilidades, siendo un Framework para la creación dinámica de mo-dernas interfaces se vale de las virtudes de HTML5, CSS3 
 y jQuery para generar potentes elementos perfectamente compatibles con los navegadores más modernos como también para los 
 dispositivos móviles más utilizados en la actualidad. [15]
 La compatibilidad de Kendo UI es uno de sus puntos fuertes, ya sea con modernos navegadores o sistemas operativos móviles:
 \begin{enumerate}
  \item Internet Explorer 7+
  \item Firefox 3+
  \item Safari 4+
  \item Chrome
  \item Opera 10+
  \item Android 2.0+
  \item iOS 3.0+
  \item BlackBerry OS 6.0+
  \item webOS 2.2+
 \end{enumerate}

 
\end{itemize}

%CoffeeScript \footnote{http://coffeescript.org/}: Es un lenguaje con sabor a Java que se puede traducir fácilmente a Java.

%jQuery (http://jquery.com/). Biblioteca en JS para manipular documento HTML basados en DOM.

%poner las metodologías que dijo el profe?
