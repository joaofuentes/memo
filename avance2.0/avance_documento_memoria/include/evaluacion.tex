\section{Evaluación}
%	Preparación de ambiente de desarrollo
%	Posibles errores y como solucionarlos en los anexos mejor.

En esta sección se analizará en profundidad, como se preparará el ambiente de desarrollo de cada herramienta elegida, ya sea 
tradicional o emergente.\\

\subsection{Tecnologías Tradicionales}

%poner las pruebas preliminares en la parte de criterios? preguntarle al profe

\subsubsection{Symfony}

Consiste en un framework de PHP, orientado a desarrollo web. Los pasos para instalar y dejar funcional esta herramienta son:

\begin{enumerate}
 \item Descargar el archivo desde \textit{http://symfony.com/download}. Se recomienda bajar la última versión 
       estable\footnote{Hasta la fecha es la versión 2.2.}.
 \item Descomprimir el archivo una vez descargado.
 \item Abrir una consola y posicionarse en la carpeta descomprimida.
 \item Seguir las instrucciones del archivo README.md:
 \begin{enumerate}
  \item Ejecutar php composer.phar install. En caso de error ejecutar como root.
  \item Ejecutar el script de chequeo para ver si está todo en orden php app/check.php
  \item En caso de haber errores, consultar en la sección de anexos.
  %php cli,php pdo php xml,
  \item Actualizar el timezone en php.ini ubicado en /etc.
 \end{enumerate}
\end{enumerate}

Para crear una aplicación de prueba:
\begin{enumerate}
 \item Seguir las instrucciones de la documentación en [40].
 \item Es posible que se deba mover la carpeta que contenga a la aplicaciónde prueba a la ruta /var/html/www
 \item En caso de error de estilo:
 \begin{verbatim}
  Change the permissions of the "app/cache/" directory 
   so that the web server can write into it.
  Change the permissions of the "app/logs/" directory 
   so that the web server can write into it.
 \end{verbatim}
 Se puede solucionar mediante [41]:
 \begin{verbatim}
  rm -rf app/cache/*
  rm -rf app/logs/* 
  chmod 777 app/cache
  chmod 777 app/logs
 \end{verbatim}

 Si bien hay soluciones que cuentan con un mayor nivel de seguridad, se escapan del alcance de esta memoria.

\end{enumerate}


\subsubsection{CakePHP}

Consiste en un framework de PHP, orientado a desarrollo web. Los pasos para instalar y dejar funcional esta herramienta son:

\begin{enumerate}
 \item Completar
 \item Descargar el archivo desde \textit{http://cakephp.org/}, en ``Download'' Se recomienda bajar la última versión 
       estable\footnote{Hasta la fecha es la versión 2.3.x.}.
 \item Descomprimir el archivo una vez descargado.% en /var/www/html.
\end{enumerate}

\subsubsection{Apache Server}

Consiste en el servidor web  \textit{ ...Completar}

\subsection{Tecnologías Emergentes}

\subsubsection{Rails}

Consiste en un framework de Ruby, orientado a desarrollo web. Los pasos para instalar y dejar funcional esta herramienta son:

\begin{enumerate}
 \item Completar
 \item Abrir una consola  y posicionarse en la carpeta descomprimida.
 \item despues de realizar las configuraciones necesarias (con rails new y eso ) \textbf{modificar y completar este punto}
 \item Acceder a app/config dentro de la instalación, y como root ejecutar rails server.
\end{enumerate}


\subsubsection{Django}

Consiste en un framework de Python, orientado a desarrollo web. Los pasos para instalar y dejar funcional esta herramienta son:
\begin{enumerate}
 \item Descargar el archivo desde \textit{https://www.djangoproject.com/download/}. Se recomienda bajar la última versión 
       estable\footnote{Hasta la fecha es la versión 1.5.1.}.
 \item Descomprimir el archivo una vez descargado.
 \item Abrir una consola  y posicionarse en la carpeta descomprimida.
 \item Ejecutar python setup.py install (como root).
 \item Para sincronzar con la base de datos Mysql, utilizando syncdb \footnote{Considerando Mysql instalado.}
 \begin{enumerate}
  \item Instalar mysql-python, en Fedora 17 yum install mysql-python.
  \item Para ver la solucion a posibles errores, puede visitar la sección de anexos.
 \end{enumerate}

Django cuenta con una serie de archivos que se crean de forma automatica a la hora de crear un proyecto, ellos son:
\begin{itemize}
 \item \textbf{manage.py}: Es una utilidad de linea de comandos que permite al desarrollador interactuar con el proyecto de diversas 
      formas.
 \item \textbf{\_\_init\_\_.py}: Es un archivo vacio que le dice a Django que, la carpeta en la que se encuentra, debe ser considerada como
      un paquete.
 \item \textbf{settings.py}: Es un archivo que contine configuraciones para el proyecto en particular.
 \item \textbf{url.py}: Es un archivo que posee una ``taba de contenidos'' de como funcionan las URL.
\end{itemize}
Para consultar información detallada, puede acceder a la documentación de Django [39].



\end{enumerate}



\subsubsection{Nodejs}

Consiste en un framework de js basado en el motor V8 de Google. Los pasos para instalar y dejar funcional esta herramienta son:

\begin{enumerate}
 \item Descargar el archivo en \textit{http://nodejs.org/}, install.
 \item Una vez descargado, descomprimir y seguir los pasos del archivo ``Readme.md''.
 \item En una consola y posicionandose en la carpeta descomprimida, ejecutar:
 \begin{enumerate}
  \item ./configure
  \item make
  \item make install (pide permisos de root)
 \end{enumerate}
 \item Es necesario tener gcc instalado, sino aparecerá un error.
 \item Para ver soluciones a posibles errores, puede visitar la sección de anexos.
\end{enumerate}

Para configurar la herramienta con la Base de Datos:
\textit{agregar informacion cuando se realice}

\subsection{Motor de Base de Datos}
 
Para instalar y configurar el motor de Base de Datos Mysql (común para todas las herramientas) en Fedora 17 [38], es necesario:
\begin{enumerate}
 \item Abrir una consola y acceder como usuario root.
 \item Ejecutar yum install mysql-server.
 \item Una vez instalado, ejecutar systemctl start mysqld.service.
 \item Ejecutar systemctl enable mysqld.service para que inicie el servicio al iniciar el SO \footnote{Sistema Operativo}.
 \item Conectarse a mysql a través de \textbf{mysql -u root} y:
 \begin{enumerate}
  \item Identificar la información inicial de la Base de Datos:
   \begin{center}
   \begin{verbatim}
    mysql> select user,host,password from mysql.user;
+------+-----------+-------------------------------------------+
| user | host      | password                                  |
+------+-----------+-------------------------------------------+
| root | localhost |                                           |
| root | tarrito   |                                           |
| root | 127.0.0.1 |                                           |
| root | ::1       |                                           |
|      | localhost |                                           |
|      | tarrito   |                                           |
+------+-----------+-------------------------------------------+


   \end{verbatim}
   \end{center}
\item Establecer contraseñas:
   \begin{center}
   \begin{verbatim}
mysql> set password for root@localhost=password('******');
mysql> set password for root@'tarrito'=password('******');
mysql> set password for root@'127.0.0.1'=password('******');
   \end{verbatim}
   \end{center}

 \item El password no necesariamente debe ser el mismo de root.
\item Y verificando la información:
   \begin{center}
   \begin{verbatim}
mysql> select user,host,password from mysql.user;
+------+-----------+-------------------------------------------+
| user | host      | password                                  |
+------+-----------+-------------------------------------------+
| root | localhost | *7C65DFE6EA91038267275D4DB8D9C16DAFCBD3F8 |
| root | tarrito   | *7C65DFE6EA91038267275D4DB8D9C16DAFCBD3F8 |
| root | 127.0.0.1 | *7C65DFE6EA91038267275D4DB8D9C16DAFCBD3F8 |
| root | ::1       |                                           |
|      | localhost |                                           |
|      | tarrito   |                                           |
+------+-----------+-------------------------------------------+
   \end{verbatim}
   \end{center}

 \end{enumerate}

 
\end{enumerate}

\newpage