CLIENTE SERVIDOR

http://bit.ly/16zWr6c (hora de consulta 13:32 17/06/2013)
Con respecto a la definición de arquitectura cliente/servidor se encuentran las siguientes definiciones:

    Cualquier combinación de sistemas que pueden colaborar entre si para dar a los usuarios toda la información que 
     ellos necesiten sin que tengan que saber donde esta ubicada.
    Es una arquitectura de procesamientos cooperativo donde uno de los componentes pide servicios a otro.
    Es un procesamiento de datos de índole colaborativo entre dos o más computadoras conectadas a una red.
    El término cliente/servidor es originalmente aplicado a la arquitectura de software que describe el procesamiento 
     entre dos o más programas: una aplicación y un servicio soportante.
    IBM define al modelo Cliente/Servidor. "Es la tecnología que proporciona al usuario final el acceso transparente a las 
     aplicaciones, datos, servicios de cómputo o cualquier otro recurso del grupo de trabajo y/o, a través de la organización, 
     en múltiples plataformas. El modelo soporta un medio ambiente distribuido en el cual los requerimientos de servicio hechos 
     por estaciones de trabajo inteligentes o "clientes'', resultan en un trabajo realizado por otros computadores llamados 
     servidores".
    "Es un modelo para construir sistemas de información, que se sustenta en la idea de repartir el tratamiento de la información y 
    los datos por todo el sistema informático, permitiendo mejorar el rendimiento del sistema global de información"
    
Evolución de la arquitectura cliente servidor

La era de la computadora central

"Desde sus inicios el modelo de administración de datos a través de computadoras se basaba en el uso de terminales remotas, 
que se conectaban de manera directa a una computadora central". Dicha computadora central se encargaba de prestar servicios 
caracterizados por que cada servicio se prestaba solo a un grupo exclusivo de usuarios.

La era de las computadoras dedicadas

Esta es la era en la que cada servicio empleaba su propia computadora que permitía que los usuarios de ese servicio se 
conectaran directamente. Esto es consecuencia de la aparición de computadoras pequeñas, de fácil uso, más baratas y más 
poderosas de las convencionales.

La era de la conexión libre

Hace mas de 10 años que la computadoras escritorio aparecieron de manera masiva. Esto permitió que parte apreciable de la 
carga de trabajo de cómputo tanto en el ámbito de cálculo como en el ámbito de la presentación se lleven a cabo desde el 
escritorio del usuario. En muchos de los casos el usuario obtiene la información que necesita de alguna computadora de servicio. 
Estas computadoras de escritorio se conectan a las computadoras de servicio empleando software que permite la emulación de algún 
tipo de terminal. En otros de los casos se les transfiere la información haciendo uso de recursos magnéticos o por trascripción.

La era del cómputo a través de redes

Esta es la era que esta basada en el concepto de redes de computadoras, en la que la información reside en una o varias computadoras, 
los usuarios de esta información hacen uso de computadoras para laborar y todas ellas se encuentran conectadas entre si. Esto brinda 
la posibilidad de que todos los usuarios puedan acceder a la información de todas las computadoras y a la vez que los diversos sistemas
intercambien información.


La era de la arquitectura cliente servidor

"En esta arquitectura la computadora de cada uno de los usuarios, llamada cliente, produce una demanda de información a cualquiera de 
las computadoras que proporcionan información, conocidas como servidores"estos últimos responden a la demanda del cliente que la produjo.

Los clientes y los servidores pueden estar conectados a una red local o una red amplia, como la que se puede implementar en una empresa 
o a una red mundial como lo es la Internet.

Bajo este modelo cada usuario tiene la libertad de obtener la información que requiera en un momento dado proveniente de una o varias 
fuentes locales o distantes y de procesarla como según le convenga. Los distintos servidores también pueden intercambiar información 
dentro de esta arquitectura.



QUE ES UN CLIENTE

Es el que inicia un requerimiento de servicio. El requerimiento inicial puede convertirse en múltiples requerimientos de trabajo a 
través de redes LAN o WAN. La ubicación de los datos o de las aplicaciones es totalmente transparente para el cliente.

QUE ES UN SERVIDOR

Es cualquier recurso de cómputo dedicado a responder a los requerimientos del cliente. Los servidores pueden estar conectados a 
los clientes a través de redes LANs o WANs, para proveer de múltiples servicios a los clientes y ciudadanos tales como impresión, 
acceso a bases de datos, fax, procesamiento de imágenes, etc.

CARACTERISTICAS DEL MODELO CLIENTE/SERVIDOR

En el modelo CLIENTE/SERVIDOR podemos encontrar las siguientes características:

1. El Cliente y el Servidor pueden actuar como una sola entidad y también pueden actuar como entidades separadas, realizando 
actividades o tareas independientes.

2. Las funciones de Cliente y Servidor pueden estar en plataformas separadas, o en la misma plataforma.

3. Un servidor da servicio a múltiples clientes en forma concurrente.

4. Cada plataforma puede ser escalable independientemente. Los cambios realizados en las plataformas de los Clientes o de 
los Servidores, ya sean por actualización o por reemplazo tecnológico, se realizan de una manera transparente para el usuario final.

5. La interrelación entre el hardware y el software están basados en una infraestructura poderosa, de tal forma que el acceso a
los recursos de la red no muestra la complejidad de los diferentes tipos de formatos de datos y de los protocolos.
6. Un sistema de servidores realiza múltiples funciones al mismo tiempo que presenta una imagen de un solo sistema a las estaciones 
Clientes. Esto se logra combinando los recursos de cómputo que se encuentran físicamente separados en un solo sistema lógico, 
proporcionando de esta manera el servicio más efectivo para el usuario final.
También es importante hacer notar que las funciones Cliente/Servidor pueden ser dinámicas. Ejemplo, un servidor puede convertirse en 
cliente cuando realiza la solicitud de servicios a otras plataformas dentro de la red.
Su capacidad para permitir integrar los equipos ya existentes en una organización, dentro de una arquitectura informática 
descentralizada y heterogénea.

7. Además se constituye como el nexo de unión mas adecuado para reconciliar los sistemas de información basados en mainframes 
o minicomputadores, con aquellos otros sustentados en entornos informáticos pequeños y estaciones de trabajo.

8. Designa un modelo de construcción de sistemas informáticos de carácter distribuido.
Su representación típica es un centro de trabajo (PC), en donde el usuario dispone de sus propias aplicaciones de oficina y 
sus propias bases de datos, sin dependencia directa del sistema central de información de la organización, al tiempo que puede 
acceder a los recursos de este host central y otros sistemas de la organización ponen a su servicio.


 TIPOS DE SERVIDOR

Servidores de archivos
Servidor donde se almacena archivos y aplicaciones de productividad como por ejemplo procesadores de texto, hojas de cálculo, etc.

Servidores de bases de datos
Servidor donde se almacenan las bases de datos, tablas, índices. Es uno de los servidores que más carga tiene.

Servidores de transacciones
Servidor que cumple o procesa todas las transacciones. Valida primero y recién genera un pedido al servidor de bases de datos.

Servidores de Groupware
Servidor utilizado para el seguimiento de operaciones dentro de la red.

Servidores de objetos
Contienen objetos que deben estar fuera del servidor de base de datos. Estos objetos pueden ser videos, imágenes, 
objetos multimedia en general.

Servidores Web
Se usan como una forma inteligente para comunicación entre empresas a través de Internet.


-----------------------------------------------------------------------------------------------------------------------------

http://bit.ly/11KJzG5 (hora de consulta 14:19 17/06/2013)

Desde el punto de vista funcional, se puede definir la computación Cliente/Servidor como una arquitectura distribuida que permite 
a los usuarios finales obtener acceso a la información de forma transparente aún en entornos multiplataforma. Se trata pues, 
de la arquitectura más extendida en la realización de Sistemas Distribuidos.

Servicio: unidad básica de diseño. El servidor los proporciona y el cliente los utiliza.
Recursos compartidos: Muchos clientes utilizan los mismos servidores y, a través de ellos, comparten tanto recursos lógicos como físicos.
Protocolos asimétricos: Los clientes inician “conversaciones”. Los servidores esperan su establecimiento pasivamente.
Transparencia de localización física de los servidores y clientes: El cliente no tiene por qué saber dónde se encuentra situado el 
recurso que desea utilizar.
Independencia de la plataforma HW y SW que se emplee.
Sistemas débilmente acoplados. Interacción basada en envío de mensajes.
Encapsulamiento de servicios. Los detalles de la implementación de un servicio son transparentes al cliente.
Escalabilidad horizontal (añadir clientes) y vertical (ampliar potencia de los servidores).
Integridad: Datos y programas centralizados en servidores facilitan su integridad y mantenimiento

El Esquema de funcionamiento de un Sistema Cliente/Servidor sería:

    El cliente solicita una información al servidor.
    El servidor recibe la petición del cliente.
    El servidor procesa dicha solicitud.
    El servidor envía el resultado obtenido al cliente.
    El cliente recibe el resultado y lo procesa.


    
Componentes de la arquitectura cliente servidor

De estas líneas se deducen los tres elementos fundamentales sobre los cuales se desarrollan e implantan los sistemas 
Cliente/Servidor: el proceso cliente que es quien inicia el diálogo, el proceso servidor que pasivamente espera a que lleguen 
peticiones de servicio y el middleware que corresponde a la interfaz que provee la conectividad entre el cliente y el servidor 
para poder intercambiar mensajes.
    
Nivel de Presentación: Agrupa a todos los elementos asociados al componente Cliente.
Nivel de Aplicación: Agrupa a todos los elementos asociados al componente Servidor.
Nivel de comunicación: Agrupa a todos los elementos que hacen posible la comunicación entre los componentes Cliente y servidor.
Nivel de base de datos: Agrupa a todas las actividades asociadas al acceso de los datos.


ELEMENTOS PRINCIPALES
CLIENTE

cliente es el proceso que permite al usuario formular los requerimientos y pasarlos al servidor. Se lo conoce con el término front-end.

Éste normalmente maneja todas las funciones relacionadas con la manipulación y despliegue de datos, por lo que están desarrollados 
sobre plataformas que permiten construir interfaces gráficas de usuario (GUI), además de acceder a los servicios distribuidos en 
cualquier parte de la red. Las funciones que lleva a cabo el proceso cliente se resumen en los siguientes puntos:

    Administrar la interfaz de usuario.
    Interactuar con el usuario.
    Procesar la lógica de la aplicación y hacer validaciones locales.
    Generar requerimientos de bases de datos.
    Recibir resultados del servidor.
    Formatear resultados.

La funcionalidad del proceso cliente marca la operativa de la aplicación (flujo de información o lógica de negocio). De este modo el 
cliente se puede clasificar en:

    Cliente basado en aplicación de usuario. Si los datos son de baja interacción y están fuertemente relacionados con la actividad de 
    los usuarios de esos clientes.
    Cliente basado en lógica de negocio. Toma datos suministrados por el usuario y/o la base de datos y efectúa los cálculos necesarios 
    según los requerimientos del usuario.

SERVIDOR

Un servidor es todo proceso que proporciona un servicio a otros. Es el proceso encargado de atender a múltiples clientes que hacen peticiones 
de algún recurso administrado por él. Al proceso servidor se lo conoce con el término back-end. El servidor normalmente maneja todas las 
funciones relacionadas con la mayoría de las reglas del negocio y los recursos de datos. Las principales funciones que lleva a cabo el 
proceso servidor se enumeran a continuación:

    Aceptar los requerimientos de bases de datos que hacen los clientes.
    Procesar requerimientos de bases de datos.
    Formatear datos para trasmitirlos a los clientes.
    Procesar la lógica de la aplicación y realizar validaciones a nivel de bases de datos.

Puede darse el caso que un servidor actúe a su vez como cliente de otro servidor. Existen numerosos tipos de servidores, cada uno de los 
cuales da lugar a un tipo de arquitectura Cliente/Servidor diferente.

    El término "servidor" se suele utilizar también para designar el hardware, de gran potencia, capacidad y prestaciones, utilizado para 
    albergar servicios que atienden a un gran número de usuarios concurrentes. Desde el punto de vista de la arquitectura cliente/servidor 
    y del procesamiento cooperativo un servidor es un servicio software que atiende las peticiones de procesos software clientes.
    
    
MIDDLEWARE

El middleware es un módulo intermedio que actúa como conductor entre sistemas permitiendo a cualquier usuario de sistemas de información 
comunicarse con varias fuentes de información que se encuentran conectadas por una red. En el caso que nos concierne, es el intermediario 
entre el cliente y el servidor y se ejecuta en ambas partes.

La utilización del middleware permite desarrollar aplicaciones en arquitectura Cliente/Servidor independizando los servidores y clientes, 
facilitando la interrelación entre ellos y evitando dependencias de tecnologías propietarias. El concepto de middleware no es un concepto 
nuevo. Los primeros * monitores de teleproceso* de los grandes sistemas basados en tecnología Cliente/Servidor ya se basaban en él, pero 
es con el nacimiento de la tecnología basada en sistemas abiertos cuando el concepto de middleware toma su máxima importancia. El middleware 
se estructura en tres niveles:

    Protocolo de transporte.
    Network Operating System (NOS).
    Protocolo específico del servicio.

Las principales características de un middleware son:

    Simplifica el proceso de desarrollo de aplicaciones al independizar los entornos propietarios.
    Permite la interconectividad de los Sistemas de Información del Organismo.
    Proporciona mayor control del negocio al poder contar con información procedente de distintas plataformas sobre el mismo soporte.
    Facilita el desarrollo de sistemas complejos con diferentes tecnologías y arquitecturas.

=============================================================================================================================================
=============================================================================================================================================

ORIENTADO A SERVICIOS
http://bit.ly/11GTGe3 (hora de consulta 15:12 17/06/2013)

Introducción   

En una empresa pueden coexistir varias aplicaciones. Esto lleva a una serie de inconvenientes que aumentan el esfuerzo y el tiempo en 
que se responde a un requerimiento determinado. Uno de los inconvenientes es, por ejemplo, ante aplicaciones diferentes probablemente 
desarrolladas en lenguajes diferentes, no poder acceder desde una de las aplicaciones hacia la otra para consultar algún dato.

Mediante la aplicación de la Arquitectura SOA se pretende solucionar los inconvenientes antes mencionados. Dentro de la arquitectura 
SOA la funcionalidad se implementa en pequeños componentes autónomos reutilizables denominados servicios [14].

SOA obtiene una integración de aplicaciones o componentes, uniendo la tecnología de información con las necesidades del negocio, 
logrando una respuesta rápida con un bajo acoplamiento, además de alcanzar como vemos en la figura 1, un ambiente operativo integrado 
que provee servicios para integrar personas, procesos e información.

--------------------------------------------------------------------------------

http://bit.ly/ZxL9OX (hora de consulta 15:24 17/06/2013)

La arquitectura orientada a servicios (SOA) no se trata de software o de un lenguaje de programación, SOA es un marco de trabajo 
conceptual que permite a las organizaciones unir los objetivos de negocio con la infraestructura de TI integrando los datos y la 
lógica de negocio de sus sistemas separados.

Desarrollada a finales de los ´90, SOA establece un marco de trabajo para servicios de red – o tareas comunes de negocios – para 
identificar el uno al otro y comunicarlo. 

xq nace??

La necesidad de tal marco se deriva de la evolución del software de negocio. En los comienzos, los desarrollos de aplicaciones 
de negocio se concentraban en necesidades específicas: contabilidad, compras, nómina de sueldos, transporte. Cada aplicación 
fue desarrollada sin consideración de otros sistemas en la empresa y como comunicarse con ellos. Porque las aplicaciones eran 
auto suficientes, la información común a toda la empresa (como por ejemplo: la dirección del cliente) y funciones específicas 
de negocios (como por ejemplo: buscar un nombre) aparecían en todas partes y requerían un código complejo para, todos o muchos 
de los sistemas independientes.

-----------------------------------------------------------------------------------------

http://bit.ly/19dWxRZ (hora de consulta 15:32 17/06/2013)


(En inglés Service Oriented Architecture o SOA), es un concepto de arquitectura de software que define la utilización de 
servicios para dar soporte a los requisitos de software del usuario.

SOA es una arquitectura de software que permite la creación y/o cambios de los procesos de negocio desde la perspectiva 
de TI de forma ágil, a través de la composición de nuevos procesos utilizando las funcionalidades de negocio que están 
contenidas en la infraestructura de aplicaciones actuales o futuras (expuestas bajo la forma de webservices).

SOA define las siguientes capas de software:

    aplicaciones básicas, sistemas desarrollados bajo cualquier arquitectura o tecnología, geográficamente dispersos y bajo 
     cualquier figura de propiedad;
    de exposición de funcionalidades, donde las funcionalidades de la capa aplicativas son expuestas en forma de servicios (webservices);
    de integración de servicios, facilitan el intercambio de datos entre elementos de la capa aplicativa orientada a procesos 
     empresariales internos o en colaboración;
    de composición de procesos, que define el proceso en términos del negocio y sus necesidades, y que varia en función del negocio;
    de entrega, donde los servicios son desplegados a los usuarios finales.

Los beneficios que puede obtener una compañía que adopte SOA son:

    Mejora en los tiempos de realización de cambios en procesos.
    Facilidad para evolucionar a modelos de negocios basados en tercerización.
    Facilidad para abordar modelos de negocios basados en colaboración con otros entes (socios, proveedores).
    Poder para reemplazar elementos de la capa aplicativa SOA sin disrupción en el proceso de negocio
    Facilidad para la integración de tecnologías disímiles

CONEXIONES

La tecnologia de servicios web es esencial para el uso de XML y asi crear una sólida relación.

La siguiente figura ilustra una base arquitectura orientada a servicios. Se muestra un servicio de los consumidores en el derecho de 
enviar un mensaje de solicitud de servicio a un proveedor de servicios a la izquierda. El proveedor de servicios devuelve un mensaje 
de respuesta con el servicio al consumidor. La solicitud y la posterior respuesta de las conexiones se definen de alguna manera que 
sea comprensible tanto para el consumidor de servicios y proveedor de servicios. ¿Cómo esas conexiones se definen se explica en Web 
Services explicó. Un proveedor de servicios también puede ser un consumidor de servicios.

---------------------------------------------------------------------------------------------------------------------------------------------

http://bit.ly/1amdWtj (hora de consulta 15:56 17/06/2013)

No existe realmente una lista de Principios definidos, por lo tanto, proporcionaremos un conjunto de Principios que están muy asociados con al orientación a Servicios.
Basándonos en Thomas Erl, estos son:



    Los Servicios deben ser reusables: Todo servicio debe ser diseñado y construido pensando en su reutilización dentro de la misma 
    aplicación, dentro del dominio de aplicaciones de la empresa o incluso dentro del dominio público para su uso masivo.

    Los Servicios deben proporcionar un contrato formal: Todo servicio desarrollado, debe proporcionar un contrato en el cual figuren: 
    el nombre del servicio, su forma de acceso, las funcionales que ofrece, los datos de entrada de cada una de las funcionalidades y 
    los datos de salida. De esta manera, todo consumidor del servicio, accederá a este mediante el contrato, logrando así la independencia 
    entre el consumidor y la implementación del propio servicio.

    Los Servicios deben tener bajo acoplamiento: Es decir, que los servicios tienen que ser independientes los unos de los otros. Para 
    lograr ese bajo acoplamiento, lo que se hará es que cada vez que se vaya a ejecutar un servicio, se accederá a él a través del contrato, 
    logrando así la independencia entre el servicio que se va a ejecutar y el que lo llama. De esta manera serán totalmente reutilizables.

    Los Servicios deben permitir la composición: Todo servicio debe ser construido de tal manera que pueda ser utilizado para construir 
    servicios genéricos de más alto nivel, el cual estará compuesto de servicios de más bajo nivel. En el caso de los Servicios Web, esto 
    se logrará mediante el uso de los protocolos para orquestación(WS-BPEL) y coreografía (WS-CDL).

    Los Servicios deben de ser autónomos: Todo Servicio debe tener su propio entorno de ejecución. De esta manera el servicio es totalmente 
    independiente y nos podemos asegurar que así podrá ser reutilizable desde el punto de vista de la plataforma de ejecución.

    Los Servicios no deben tener estado: Un servicio no debe guardar ningún tipo de información. Esto es así porque una aplicación está 
    formada por un conjunto de servicios, lo que implica que si un servicio almacena algún tipo de información, se pueden producir 
    problemas de inconsistencia de datos. La solución, es que un servicio sólo contenga lógica, y que toda información esté almacenada 
    en algún sistema de información sea del tipo que sea.

    Los Servicios deben poder ser descubiertos: Todo servicio debe poder ser descubierto de alguna forma para que pueda ser utilizado, 
    consiguiendo así evitar la creación accidental de servicios que proporcionen las mismas funcionalidades. En el caso de los Servicios 
    Web, el descubrimiento se logrará publicando los interfaces de los servicios en registros UDDI.
    
=============================================================================================================================================
=============================================================================================================================================+

MODELO VISTA CONTROLADOR

http://bit.ly/10eAXul(hora de consulta 16:05 17/06/2013)

Es un patrón de arquitectura de las aplicaciones software
Separa la lógica de negocio de la interfaz de usuario
Facilita la evolución por separado de ambos aspectos
Incrementa reutilización y flexibilidad

(tiene un ejemplo practico en java)

----------------------------------------------------------------------------------------------------------------------------------------------

http://bit.ly/l8dMhw (hora de consulta 16:13 17/06/2013)


El Model-View-Controller (Modelo-Vista-Controlador, en adelante MVC) fue introducido inicialmente en la comunidad de desarrolladores
de Smalltalk-80


MVC divide una aplicación interactiva en 3 áreas: procesamiento, salida y entrada. Para esto, utiliza las siguientes abstracciones:

    Modelo (Model): Encapsula los datos y las funcionalidades. El modelo es independiente de cualquier representación de salida y/o 
     comportamiento de entrada.
    Vista (View): Muestra la información al usuario. Pueden existir múltiples vistas del modelo. Cada vista tiene asociado un componente 
     controlador.
    Controlador (Controller): Reciben las entradas, usualmente como eventos que codifican los movimientos o pulsación de botones del 
     ratón, pulsaciones de teclas, etc. Los eventos son traducidos a solicitudes de servicio ("service requests") para el modelo o la vista.

-------------------------------------------------------------------------------------------------------------------------------------------


http://bit.ly/11k1vuJ (hora de consulta 16:35 17/06/2013)

¿Qué es y en donde se utiliza más frecuentemente el Modelo Vista Controlador?

Modelo Vista Controlador (MVC) es un patrón de arquitectura de software que separa los datos de una aplicación, la interfaz de usuario, 
y la lógica de control en tres componentes distintos. El patrón MVC se ve frecuentemente en aplicaciones web, donde la vista es la 
página HTML y el código que provee de datos dinámicos a la página, el modelo es el Sistema de Gestión de Base de Datos y la Lógica de 
negocio y el controlador es el responsable de recibir los eventos de entrada desde la vista

