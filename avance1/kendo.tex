%En este apartado se verán las características de 


Kendoui es una heramienta basada en jquery (que a su vez está basada en js) html5 y css3.

Según [1], cuenta con:
\begin{enumerate}
 \item  Web app developmet framework
 \item Mobile app developmnet framwork.
 \item Data viz, para poder ver datos, gráficos
\end{enumerate}


%como idea usar esto para hacer una presentancion de memoria en html5 y el data viz de las
%herramientas en kendo$

``Kendo UI es todo lo que una persona necesita para construir sitos web y aplicaciones para dispositivos móviles
en HTML5. Hoy en día la productividad para los desarrolladores HTML/JQuery es baja``. Con estas palabras se describe
esta herramienta en su página oficial.

Kendo UI lo tiene todo: Widgets basados en JQuery, framework basado en el ''Patrón Modelo-Vista-Modelo de Vista``,
temas, templates y varias cosas más.

Una de las caracteristicas interesantes, es que está listo
para trabajar en pantallas táctiles, quizás orientado a la integracion de tablets y netbooks.\\

Destaca su compatibilidad con fiferox, opera, chrome, safari, IE.\\

Kendo UI incorpora también varias características interesantes para el desarrollo de aplicaciones móviles,
por ejemplo esta listo para aceptar acciones Touch, imprescindibles en la creación de proyectos que van a 
ser utilizados desde tablets, como así también en muchos de los móviles de hoy en día.[4]\\

La compatibilidad de Kendo UI es uno de sus puntos fuertes.\\

Una interesante página de comparacion entre jquery y kendo [6]

Cuenta con versiones de pago, prueba por 30 dias y open source

\subsection{Plataformas}
Las plataformas compatibles con kendoui son:

\begin{itemize}
 \item Fiferox
 \item Opera
 \item Chrome
 \item Safari
 \item IE
\end{itemize}

%Mirando [3], se me ocurre hacer un sitio con  kendo y nodejs en el server

[1] http://www.kendoui.com(hora de consulta 29-11-12 01:36)

[2] http://www.kendoui.com/web.aspx (hora de consulta 29-11-12 01:42)

[3] http://demos.kendoui.com/web/overview/index.html (hora de consulta 29-11-12 01:49)

[4] http://www.kabytes.com/programacion/kendo-ui-framework-html5-css3-y-jquery/ (hora de consulta 29-11-12 01:51)

[5] http://docs.kendoui.com/ (hora de consulta 29-11-12 01:55)

[6] http://jqueryuivskendoui.com/ (hora de consulta 29-11-12 01:58)


\subsection{Hello World}
Es necesario registrarse en la página para poder tener acceso a las descargas. [7]\\

Una vez descargado, es necesario descomprimir e ingresar a la carpeta examples/web. Existen variados ejemplos
y se utilizan de forma similar  a jquery, es decir, se invocan a través de:

\begin{verbatim}
   <script src="../../../js/jquery.min.js"></script>
   <script src="../../../js/kendo.web.min.js"></script>
\end{verbatim}

Dentro de las descargas se incluyen  

[7] http://www.kendoui.com/download.aspx (hora de consulta 01-12-12 17:10)