%Este apartado corresponde a la información recapitulada del 
%motor javascript v8:

¿Qué es el V8? El motor V8 JavaScript es el motor JavaScript subyacente que Google usa con su navegador
Chrome. 
Pocas personas piensan en lo que en realidad sucede con JavaScript en el cliente. Bien, un motor 
JavaScript en realidad interpreta el código y lo ejecuta. Con el V8, Google creó un intérprete ultra-rápido 
escrito en C++, con otro aspecto único: se puede descargar el motor e incorporarlo a cualquier aplicación escrita en c++. 
No está restringido a ejecutarse en un navegador. Así, Node en realidad usa el motor V8 JavaScript 
escrito por Google y le da otro propósito para usarlo en el servidor. [5]

Dentro de sus principales características se pueden ennumerar[3]:

\begin{enumerate}
 \item acceso rápido a las propiedades de los objetos, a través de la creación y uso de clases invisibles en modo "background"
(explicar el ejemplo de los puntos)
 \item Generación dinámica de código de máquina, a través de métodos predictivos de variables, en los que en caso de no acertar, 
 este código generado es parchado "inline" de acuerdo a los cambios que se vayan generando. (explicar con un ejemplo mejor)
 \item colector de basura eficiente[4], V8 libera la memoria de los objetos que ya no son utilizados en un proceso conocido como
 recolección de basura. Con el fin de que este proceso sea eficiente y acertado, v8:
    \begin{enumerate}
     \item Detiene la ejecución de los programas cuando comienza el ciclo de recolección de basura (lo hace varias veces)
     \item Procesa solo aquellas partes que quedan en la categoría de basura la mayor parte de los ciclos. 
     \item Sabe exactamente donde estan los punteros (en memoria), lo cual evita falsos positivos que 
    puedan desencadenar en problemas de alocación memoria.
    \end{enumerate}
\end{enumerate}

%1) acceso rápido a a las propiedades de los objetos, a través de la creación y uso de clases invisibles en modo "background"
%(explicar el ejemplo de los puntos)

%2) Generación dinámica de código de máquina, a través de métodos predictivos de variables,en los que en caso de no acertar, este código generado es parchado "inline"
%de acuerdo a los cambios que se vayan generando. (explicar con un ejemplo mejor)

%Al agregar atributos, no se compila todo de nuevo,  sino que se van cambiando los trozos de código entremedio (o almenos eso entiendo del párrafo anterior)
%En caso que exista una coincidencia, solo se retorna el valor. Esta ventaja es más notoria cuando hay muchas  instancias de un objeto que desea acceder a las clases ocultas.


%3) colector de basura eficiente[4]V8 reclaims memory used by objects that are no longer required in a process known as garbage 
%ollection. To ensure fast object allocation, short garbage collection pauses, and no memory fragmentation V8 employs a stop-the-world, generational, accurate, garbage collector. This means that V8:

% a. Detiene a ejecucon de los programas cuando comienza el ccilco de recoleccion de basura
% b. Procesa solo aquellas partes que quedan en la categrria d basura la mayor parte de los ciclos?
% c. Sabe exactamente donde estan los punteros (en memoria), lo cual evita falsos positivos que puedan desencadenar en problemas de memoria.

V8 está escrito en C++ y corresponde a un motor de javascript que puede correr de forma "standalone" o embebido 
en aplicaciones escritas en C++.\\


desde el punto de vista del diseño cuando se emebeben aplicaciones.
    
\begin{itemize}
 \item A handle is a pointer to an object. All V8 objects are accessed using handles, they are necessary because
   of the way the V8 garbage collector works.
 \item A scope can be thought of as a container for any number of handles. When you've finished with your handles,
   instead of deleting each one individually you can simply delete their scope.
 \item A context is an execution environment that allows separate, unrelated, JavaScript code to run in a single
   instance of V8. You must explicitly specify the context in which you want any JavaScript code to be run.
\end{itemize}



Por otro lado, para una desmostración de como correr v8 en modo "standalone", dentro de la carpeta samples, existe un shell 
escrito en c++[2], el cual permite implementar una terminal simple de javascript
%$$(probablemente a través de esto se puede correr javascript en el lado del server, usando consolas )$$
se compila a través de:
\begin{verbatim}
g++ -Iinclude samples/shell.cc -o shell-v8 
out/x64.release/obj.target/tools/gyp/libv8\_{base,snapshot}.a -lpthread
\end{verbatim}

%NOTA: hasta ahora se copila siempre igual, cambiando la ruta del archivo a compilar


Al parecer es posible utilizarlo con php [6], utilizando modulos de integración de php.

\subsection{Plataformas}
 Corresponde a un motor de carácter opensource capaz de correr en:
\begin{enumerate}
 \item Windows (desde xp en adelante)
 \item Mac OS X (desde 10.5 en adelante)
 \item Sistemas Linux (IA-32, x64)
 \item Procesadores ARM 
\end{enumerate}


Bibliografía asociada

[1] https://developers.google.com/v8/desing (hora de consulta 26-11-12 18:17)

[2] http://stackoverflow.com/questions/1802478/running-v8-javascript-engine-standalone (hora de consulta 27-11-12 00:23)

[3] http://www.youtube.com/watch?v=hWhMKalEicY (hora de consulta 18-11-12 17:35)

[4] https://developers.google.com/v8/embed (hora de consulta 27-11-12 00:38)

[5] http://www.ibm.com/developerworks/ssa/opensource/library/os-nodejs/index.html (hora de consulta 28-11-12 01:08)

[6] http://php.net/manual/es/book.v8js.php (hora de consulta 28-11-12 02:57)


\subsection{Hello World}

\begin{itemize}
 \item Para poder descargar esta herramienta es necesario entrar a esta página y seguir las instrucciones:
		http://code.google.com/p/v8/wiki/UsingGit
 \item  Una vez clonado el repositorio, seguir las instrucciones de 
	  http://code.google.com/p/v8/wiki/BuildingWithGYP
 \item 	hacer make x64 en vez de ia32 (debido a q uso fedora 64b) esperar muuuuuuucho!!
 \item  Seguir las instruciones para el hello world (embebido en c++): 
	  https://developers.google.com/v8/get\_started
 \item Para compilar
 \begin{verbatim}
    g++ -Iinclude <ruta de archivo a compilar> -o shell-v8 
    out/x64.release/obj.target/tools/gyp/libv8_{base,snapshot}.a -lpthread
 \end{verbatim}

 
\end{itemize}
