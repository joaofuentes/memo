\section{Introducción}

La aparición de aplicaciones y sitios Web ha resultado en la creación y explotación de nuevos 
mercados conformados por servicios con los que antes sólo se podía soñar, algunos ejemplos son: sitios de 
comercio electrónico, el \textit{E-learning} 
%http://www.imh.es/elearning-es/que-es-elearning?language_sync=1
\footnote{El E-learning es un modelo de formación a distancia que utiliza Internet como herramienta de aprendizaje. 
Este modelo permite al alumno realizar el curso desde cualquier parte del mundo y a cualquier hora.
Con un ordenador y una conexión a Internet, el alumno realiza las actividades interactivas planteadas, accede a toda 
la información necesaria para adquirir el conocimiento, recibe ayuda del profesor; se comunica con su tutor y sus compañeros, o 
evalúa su progreso.[1]}; redes sociales, incluyendo sus múltiples servicios de mensajería, juegos online y actualización 
de contenido en tiempo real; sitios de \textit{streaming} 
\footnote{El término streaming hace alusión a una corriente continua, en este caso, de datos.
Streaming corresponde a  la distribución de contenido multimedia a través de una red (generalmente internet)
de tal forma que el usuario consume el producto al mismo tiempo que se va descargando (por ejemplo Youtube). 
Este tipo de tecnología funciona mediante un búfer de datos que va almacenando el contenido descargado para 
luego mostrarse al usuario; a diferencia de la descarga de archivos, que requiere que el usuario descargue los 
archivos por completo para poder acceder a ellos.}, tan sólo por nombrar algunas [12]\footnote{Página 1}. Esto conlleva a un importante 
crecimiento y evolución tanto de la metodología desarrollo como de las tecnologías utilizadas a la hora de crear 
una aplicación web.
%sacado de procesos agiles para el desarrollo de aplicaciones web
%Paloma Cáceres, Esperanza Marcos Grupo Kybele Departamento de Ciencias Experimentales e Ingeniería Universidad Rey Juan Carlos
%C/ Tulipán, s/n, 28933 – Móstoles, Madrid (España) {p.caceres/cuca}@escet.urjc.es


El sitio Web es el medio más barato para darse a conocer rápidamente con un alcance mundial. 
Esto es extensible no sólo a empresas que comercializan productos y servicios, o a profesionales 
autónomos, sino que también a personas u organizaciones que actúan sin ánimo de lucro, que intentan
divulgar sus obras, inquietudes o ideas.

Al comienzo, los sitios Web ofrecían casi de forma exclusiva contenidos basados en texto 
y eran bastante estáticos; en la actualidad son sitios interactivos con abundancia de elementos multimedia.
%Por lo que queda en manifiesto la constante 
%sacado de http://www.ramonmillan.com/tutoriales/dhtml.php

Como se sabe, la evolución tecnológica se hace presente en todas y cada una de las áreas de investigación, 
tanto en las ciencias de la física y química que permiten la construcción de hardware más potente, como en 
los procesos y tecnologías de desarrollo de software, específicamente, y lo que es de interés para este documento, 
las tecnologías de Desarrollo Web.

%El propósito de este artículo es comparar y detectar cuáles son las deficiencias que poseen dichas
%tecnologías. Por otro lado se busca determinar si alguna de estas tecnologías puede condicionar la metodología
%a utilizar durante el desarrollo de la aplicación. (ESTO ES PARA LA MEMORIA, NO PARAEL AVANCE)

El objetivo de este documento es hacer hincapié y convencer al lector sobre la importancia de investigar nuevas
tecnologías, que en este caso corresponden a las de desarrollo web.

Para lograrlo, el documento se estructura de la siguien-te forma: se explicará que es una aplicación web y que métodos/procesos
de desarrollo se utilizan en su construcción. Posteriormente se hará un barrido histórico respecto a las tecnologías utilizadas
a la hora de crear este tipo de aplicación, partiendo desde la llamada web 1.0 hasta cubrir términos como web semántica; haciendo
hincapié en las situaciones y problemas que han gatillado la necesidad de crear, desarrollar, investigar y probar nuevas herramientas,
es decir la evolución inherente al ser humano. Posterior a este apartado se presentan los objetivos propuestos en la memoria de pregrado,
la posible métodología de trabajo que será utilizada para lograr dichos objetivos, y finalmente, las conclusiones del autor respecto a 
la investigación realizada, además de la bibliografía consultada para la realización de este documento.

