La idea de este documento es poder dejar en claro los pasos necesarios para obtener,
instalar y dar los primeros pasos de la diversas herramientas que se estudiarán en esta 
memoria de pregrado.

Estas pruebas se realizarán en un sistema operativo GNU/Linux (específicamnete Fedora 17 (64b)).

Además se hará una breve reseña.

El orden establecido es:

  1. V8 Engine
  2. Node.js
  3. QOOXDOO
  4. Jquery
  5. Kendo



1. V8: Consiste en un motor de js creado por google, y escrito en c++. Puede ser utilizado 
   en aplicaciones web, o de forma "standalone" en aplicaciones c++.

   1.1 Para poder descargar esta herramienta:
   
   1.1.1 Es necesario entrar a esta página y seguir las instrucciones:
		http://code.google.com/p/v8/wiki/UsingGit
   1.1.2 Una vez clonado el repo, seguir las instrucciones de http://code.google.com/p/v8/wiki/BuildingWithGYP
	hacer make x64 en vez de ia32 (debido a q uso fedora 64b)
        esperar muuuuuuucho!!
     
   1.1.3 Seguir las instruciones para el hello world en https://developers.google.com/v8/get_started
Para compilar       
g++ -Iinclude <ruta de archivo a compilar> -o shell-v8 out/x64.release/obj.target/tools/gyp/libv8_{base,snapshot}.a -lpthread
 
    A handle is a pointer to an object. All V8 objects are accessed using handles, they are necessary because 
   of the way the V8 garbage collector works.
    A scope can be thought of as a container for any number of handles. When you've finished with your handles, 
   instead of deleting each one individually you can simply delete their scope.
    A context is an execution environment that allows separate, unrelated, JavaScript code to run in a single 
   instance of V8. You must explicitly specify the context in which you want any JavaScript code to be run.

 
   Información:
	Handles and Garbage Collection
	


2. Node.js: Consiste en un framework de js basado en el motor V8

  2.1 Para poder descargar esta herramienta:
 
  2.1.1 Es necesario entrara a esta página http://nodejs.org/, install, y luego seguir los pasos
        que salen en el archivo README.md

        ./configure
     	make
  	make install (pide permisos de root)

	para el error del gcc 
	
make
make -C out BUILDTYPE=Release V=1
make[1]: Entering directory `/home/joao/memoria-jfuentes-heramientas/nodejs/node-v0.8.14/out'
  g++ '-D_LARGEFILE_SOURCE' '-D_FILE_OFFSET_BITS=64' '-DENABLE_DEBUGGER_SUPPORT' '-DV8_TARGET_ARCH_X64' -I../deps/v8/src  -Wall -pthread -m64 -fno-strict-aliasing -O2 -fno-strict-aliasing -fno-tree-vrp -fno-rtti -fno-exceptions -MMD -MF /home/joao/memoria-jfuentes-heramientas/nodejs/node-v0.8.14/out/Release/.deps//home/joao/memoria-jfuentes-heramientas/nodejs/node-v0.8.14/out/Release/obj.target/v8_base/deps/v8/src/accessors.o.d.raw  -c -o /home/joao/memoria-jfuentes-heramientas/nodejs/node-v0.8.14/out/Release/obj.target/v8_base/deps/v8/src/accessors.o ../deps/v8/src/accessors.cc
make[1]: g++: Command not found
make[1]: *** [/home/joao/memoria-jfuentes-heramientas/nodejs/node-v0.8.14/out/Release/obj.target/v8_base/deps/v8/src/accessors.o] Error 127
make[1]: Leaving directory `/home/joao/memoria-jfuentes-heramientas/nodejs/node-v0.8.14/out'
make: *** [node] Error 2

	yum install gcc-c++




3. QOOXDOO




4. Jquery



5. Kendo
para bajar kendo hay que registrarse en[1], es posible descargar un trial de 30 días.
[1]http://www.kendoui.com/download/download-kendo.aspx?pid=1035&lict=1 (hora de consulta 29-11-12 12:55)



