\chapter{Capítulo 1}
\section{Introducción}

\subsection{Contexto}

La aparición de aplicaciones y sitios Web ha resultado en la creación y explotación de nuevos 
mercados conformados por servicios con los que antes sólo se podía soñar, algunos ejemplos son: sitios de 
comercio electrónico, el \textit{E-learning} \footnote{El E-learning es un modelo de formación a distancia 
que utiliza Internet como herramienta de aprendizaje.Este modelo permite al alumno realizar el curso desde 
cualquier parte del mundo y a cualquier hora.Con un ordenador y una conexión a Internet, el alumno realiza 
las actividades interactivas planteadas, accede a toda la información necesaria para adquirir el conocimiento, 
recibe ayuda del profesor; se comunica con su tutor y sus compañeros, o evalúa su progreso.[1]}, redes sociales, 
incluyendo sus múltiples servicios de mensajería, juegos online y actualización de contenido en tiempo real; 
sitios de \textit{streaming} \footnote{El término streaming hace alusión a una corriente continua, en este caso, de datos.
Streaming corresponde a  la distribución de contenido multimedia a través de una red (generalmente internet)
de tal forma que el usuario consume el producto al mismo tiempo que se va descargando (por ejemplo Youtube). 
Este tipo de tecnología funciona mediante un búfer de datos que va almacenando el contenido descargado para 
luego mostrarse al usuario; a diferencia de la descarga de archivos, que requiere que el usuario descargue los 
archivos por completo para poder acceder a ellos.},tan sólo por nombrar algunas [12]\footnote{Página 1}. Esto 
conlleva a un importante crecimiento y evolución tanto de la metodología desarrollo como de las tecnologías 
utilizadas a la hora de crear una aplicación web.\\


El sitio Web es el medio más barato para darse a conocer rápidamente con un alcance mundial. 
Esto es extensible no sólo a empresas que comercializan productos y servicios, o a profesionales 
autónomos, sino que también a personas u organizaciones que actúan sin ánimo de lucro, que intentan
divulgar sus obras, inquietudes o ideas.\\

Al comienzo, los sitios Web ofrecían casi de forma exclusiva contenidos basados en texto 
y eran bastante estáticos; en la actualidad son sitios interactivos con abundancia de elementos multimedia.\\


Como se sabe, la evolución tecnológica se hace presente en todas y cada una de las áreas de investigación, 
tanto en las ciencias de la física y química que permiten la construcción de hardware más potente, como en 
los procesos y tecnologías de desarrollo de software, específicamente, y lo que es de interés para este documento, 
las tecnologías de Desarrollo Web.\\

%\subsection{Motivación}

\subsection{Definición del Problema}

Hoy en día las herramientas Web son fundamentales en todo ámbito de negocios, desde soluciones comerciales
como carritos de compra, catálogos empresariales, e incluso complejos sistemas \textit{Enterprise Resource
Planning} o ERP por sus siglas en inglés, que consisten en una arquitectura de software para empresas que 
facilita e integra la información entre las funciones de manufactura,logística, finanzas y recursos humanos 
de la empresa.\\

Como se mencionó anteriormente, a la evolución de los servicios ofrecidos por las diversas empresas qwue pertencen
al rubro, hay que sumar el continuo aporte y empoderamiento de los propios usuarios quienes crean y consumen
contenido a tasas cada vez más altas.\\

Como ya ha ocurrido con la evolución desde páginas web estáticas a sitios web dinámicos, o la aparación de tecnologías
asincrónicas, es natural esperar que esta evolución continue, después de todo la web evoluciona de acuerdo a las 
necesidades del usuario.\\

Es fundamental reconocer y probar las nuevas tecnologías que han ido apareciendo con el transcurrir de los años,
de ese modo, poder evaluar de forma concreta y objetiva, en que contexto son de utilidad.\\


\subsection{Objetivos}

\subsubsection{Objetivos Generales}
%separar en general y especifico (preguntarle al profe)!!!

El objetivo de esta memoria es realizar una evaluación de diversas tecnologías emergentes de desarrollo web
y compararlas versus una implementación de carácter tradicional bajo ciertos criterios. Para ello se deben cumplir 
los siguientes objetivos generales:

\begin{enumerate}
 \item Explicar el avance de las tecnologías de desarrollo web, y cómo han evolucionado desde páginas web
	estáticas hasta el concepto de aplicaciones web.
 \item Evidenciar empíricamente la diferencia de rendimiento entre tecnologías emergentes, utilizando como
	base de comparación, tecnología de carácter más tradicional.
% \item Obtener conclusiones y proponer sugerencias sobre la implementación de 
\end{enumerate}


\subsubsection{Objetivos Específicos}
\begin{enumerate}
 \item Investigar tecnologías existentes y emergentes para el desarrollo de aplicaciones web.
 \begin{enumerate}
   \item  Revisar evolución histórica de las tecnologías asociadas a desarrollo de aplicaciones Web.
   \item  Realizar un levantamiento del estado del arte.
 \end{enumerate}
 \item Investigar y evaluar cómo las tecnologías recientes pueden ayudar a desarrollar mejores aplicaciones 
	Web y  mejorar los procesos de desarrollo de software.
 \begin{enumerate}
   \item Definir criterios de clasificación general y selección de las tecnologías específicas a investigar
   \item Aplicar los criterios anteriores para seleccionar un conjunto de tecnologías a investigar en mayor profundidad.
   \item Evaluar en más detalle las tecnologías seleccionadas.
 \end{enumerate}
 \item Definir una aplicación de prueba y desarrollar un prototipo aplicando las tecnologías seleccionadas que permita 
	entender, integrar y evaluar estas tecnologías. Comparar con la aplicación de tecnologías más tradicionales.
 \begin{enumerate}
   \item Diseñar y desarrollar el prototipo aplicando diferentes tecnologías.
   \item Evaluar cada uno de los casos de aplicación
 \end{enumerate}
 \item Definir el tipo pruebas que midan rendimiento y como se aplicarán a los prototipos. 
 \item Hacer una evaluación global y obtener conclusiones.
\end{enumerate}


\subsection{Metodología de trabajo}
 
El desarrollo de este trabajo se divide en etapas, las cuales son:
\begin{enumerate}
 \item Definición de Objetivos y Estado del Arte: Esta etapa consiste en la definición de los Objetivos Generales 
 y Específicos de esta memoria, se investiga la evolución histórica de las diversas tecnologías de desarrollo web, 
 y se realiza un levantamiento del estado del arte algunas de las metodologías más importantes de desarrollo web.
 
 \item Selección de Tecnologías y Ambiente de Desarrollo: Esta etapa consiste en la definición y
 aplicación de criterios de selección que permitan filtrar y escoger de entre las tecnologías investigadas en la etapa 
 anterior. Por otra parte costa de la implementación de un ambiente de desarrollo necesario para realizar las diversas 
 pruebas requeridas en los objetivos. Dicha implementación es necesaria para cada tecnología escogida.
 
 \item Definición de pruebas y Experimentos: Esta etapa consiste en la definición las pruebas orientadas a medir el 
 rendimiento de los prototipos construidos. Una vez definidas, se procederá a correr los experimentos en los prototipos, 
 midiendo su desempeño.
 
 \item Analisis y Resultados: En esta etapa se analizarán y compararán los resultados obtenidos en la etapa anterior,
 explicando cada uno de ellos.
 
 \item Conclusiones: La última etapa consta de la obtención de conclusiones del trabajo realizado, tomando
 en cuenta todas las etapas anteriores.
\end{enumerate}

\subsection{Estructura del documento}

En el capítulo 1 del presente documento se realiza una descripción del problema, contextualizando 
al lector y motivandolo con las consecuencias de su resolución. Además se presentan los objetivos 
que busca cumplir esta memoria.\\

En el capítulo 2 se define que es una aplicación web, y se explican los procesos de desarrollo de 
software necesarios para poder construirla; comenzando por los procesos de desarrollo de software 
en general, los cuales son los precursores de los procesos de desarrollo web.\\
%llenar
%agregar kanban?

En el capítulo 3 se realiza una revisión sobre la evolución histórica de las tecnologías web.
Se explican las características y las tecnologías utilizadas más populares en su época.\\
%llenar
% agregar ror y django de todos modos

En el capítulo 4 se definen los criterios de selección bajo los cuales se escogerán las tecnologías a 
usar en el prototipo, ampliando la investigación de las tecnologías seleccionadas.\\
%php apache// express(js) nodejs // django y ror quedan en espera por ahora.

En el capítulo 5  se definen las pruebas y experimentos a realizar utilizando las tecnologías seleccionadas, 
se describe la preparación de los ambientes de desarrollo necesarios para poder realizar dichas pruebas y
se deja en manifiesto cualquier posible problema y su respectiva solución a la hora de preparar el
ambiente de desarrollo.\\

En el capítulo 6 se ejecutan las pruebas, se reunen, analizan y comparan los resultados obtenidos.\\

El capítulo 7  está dedicado a las conclusiones obtenidas en esta memoria. Se responderá a los
objetivos planteados en el capítulo 1.\\


\newpage