\section{Bibliografía}


[1]http://www.imh.es/elearning-es/que-es-elearning hora de consulta 03:25 17/10/2012

[2]Propuestas metodológicas para el desarrollo de aplicaciones Web: una evaluación según la ingeniería de métodos
   M. Mendoza* y J. Barrios**
   Postgrado en Computación,
   Universidad de Los Andes, Facultad de Ingeniería,
   Escuela de Ingeniería de Sistemas,
   Mérida, Venezuela.
   *nella\_1@hotmail.com,
   **ijudith@ing.ula.ve
   Revista Ciencia e Ingeniería. Vol. 25 No. 2. 2004 

[3]http://www.koala-soft.com/de-web-10-a-web-30 hora de consulta 17:00 17/10/2012

[4] http://www.quantium.com.mx/2012/08/06/nodejs/ (hora de consulta 18:18 28/11/2012)

[5]La evolución de los servicios de referencia digitales en la Web 2.0
   Catuxa Seoane García
   catuxa@gmail.com
   Documentalista. Bibliotecas Municipales de A Coruña
   Vanesa Barrero Robledo
   uveybe@gmail.com
   Documentalista. Yahoo España
   Autoras del blog Deakialli DocuMental (http://www.deakialli.com)

[6]http://www.oreillynet.com/oreilly/tim/news/2005/09/30/what-is-web-20.html  hora de consulta 18:10 17/10/2012

[7]http://www.ecured.cu/index.php/Qooxdoo hora de consulta 02:45 22/10/2012

[8]http://www.desarrolloweb.com/articulos/que-es-html.html hora de consulta 22:37 17/10/2012

[9]http://www.monografias.com/trabajos7/html/html.shtml hora de consulta 22:44 17/10/2012

[10] https://npmjs.org (hora de consulta 18:10 28/11/2012)

[11]http://www.monografias.com/trabajos62/sistemas-informacion-web/sistemas-informacion-web2.shtml hora de consulta 01:59 18/10/2012

[12]Procesos Ágiles Para el Desarrollo de Aplicaciones Web 
    Paloma Cáceres, Esperanza Marcos
    Grupo Kybele
    Departamento de Ciencias Experimentales e Ingeniería
    Universidad Rey Juan Carlos
    C/ Tulipán, s/n, 28933 – Móstoles, Madrid (España)
    {p.caceres/cuca}@escet.urjc.es

[13]http://www.genbetadev.com/bases-de-datos/una-introduccion-a-mongodb hora de consulta 03:29 22/10/2012

[14]I Congreso Internacional de Ciberperiodismo y Web 2.0. Bilbao: Noviembre 2009
    ¿Web 2.0, Web 3.0 o Web Semántica?: El impacto en los sistemas de
    información de la Web
    Por Lluís Codina
    Universidad Pompeu Fabra
    www.lluiscodina.com | www.lluiscodina.com/diagramas/htm
    Página 4
    
[15]http://www.kabytes.com/programacion/kendo-ui-framework-html5-css3-y-jquery/ hora de consulta 03:35 22/10/2012

[16]http://interfacemindbraincomputer.wetpaint.com/page/2.A.1.1.7.-Computacion+Ubicua+o+Pervasiva+e+Inteligencia+ambiental hora de consulta 
21:22 24/10/2012

[17] http://nodejs.org/download/ (hora de consulta 19:10 29/11/2012) 

[18] http://slidesha.re/Yl7RJH (hora de consulta 15:39 10/04/2013) 

[19] http://book.mixu.net/ch13.html (hora de consulta 16:55 05/04/2013)

[20] http://www.ecured.cu/index.php/Arquitectura\_de\_software (hora de consulta 16:32 07/03/2013)

[21] http://sg.com.mx/content/view/922 (hora de consulta 16:42 07/03/2013)

[22] http://bit.ly/11CxHEO (hora de consulta 16:02 04/06/2013)

[23] http://davidjguru.com/2010/02/08/la-arquitectura-de-tres-capas-introduccion/ (hora de consulta 13:46 04/06/2013)

[24] http://bit.ly/11GTGe3 (hora de consulta 15:12 17/06/2013)

[25] http://bit.ly/ZxL9OX (hora de consulta 15:24 17/06/2013)

[26] http://bit.ly/19dWxRZ (hora de consulta 15:32 17/06/2013)

[27] http://bit.ly/11sQ1VT (hora de consulta 16:35 21/06/2013)

[28] http://bit.ly/1amdWtj (hora de consulta 15:56 17/06/2013)

[29] http://bit.ly/14nHu5g (hora de consulta 17:03 21/06/2013)
