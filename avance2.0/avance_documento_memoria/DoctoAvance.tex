
%BUSCAR ALGUN TEMPLATE PARA LA MEMORIA 
\documentclass[letter, 11pt]{article}
\usepackage[utf8]{inputenc}
\usepackage[spanish]{babel}
\usepackage{amsfonts}
\usepackage{amsmath}
\usepackage[pdftex]{graphicx}
\usepackage{url}
\usepackage[top=3cm,bottom=3cm,left=3.5cm,right=3.5cm,footskip=1.5cm,headheight=1.5cm,headsep=.5cm,textheight=3cm]{geometry}



\begin{document}
\title{Evaluación de tecnologías Emergentes para el desarrollo de aplicaciones web }
\author{Jo\~ao Fuentes Pacheco, Raúl Monge}
\date{\today}
\maketitle

\newpage
%\makeindex 
\newpage
%Agradecimientos
\section{Agradecimientos}

\newpage

%Resumen
\section{Resumen}

\newpage



%Introducción
%	Contexto
%	Motivación
%		Definición de problema
%		Definir la importancia de realizar dicha evaluación
%	Objetivos
%	Metodologìa de la solución
%	Estructura del documento
\chapter{Capítulo 1}
\section{Introducción}

\subsection{Contexto}

La aparición de aplicaciones y sitios Web ha resultado en la creación y explotación de nuevos 
mercados conformados por servicios con los que antes sólo se podía soñar, algunos ejemplos son: sitios de 
comercio electrónico, el \textit{E-learning} \footnote{El E-learning es un modelo de formación a distancia 
que utiliza Internet como herramienta de aprendizaje.Este modelo permite al alumno realizar el curso desde 
cualquier parte del mundo y a cualquier hora.Con un ordenador y una conexión a Internet, el alumno realiza 
las actividades interactivas planteadas, accede a toda la información necesaria para adquirir el conocimiento, 
recibe ayuda del profesor; se comunica con su tutor y sus compañeros, o evalúa su progreso.[1]}, redes sociales, 
incluyendo sus múltiples servicios de mensajería, juegos online y actualización de contenido en tiempo real; 
sitios de \textit{streaming} \footnote{El término streaming hace alusión a una corriente continua, en este caso, de datos.
Streaming corresponde a  la distribución de contenido multimedia a través de una red (generalmente internet)
de tal forma que el usuario consume el producto al mismo tiempo que se va descargando (por ejemplo Youtube). 
Este tipo de tecnología funciona mediante un búfer de datos que va almacenando el contenido descargado para 
luego mostrarse al usuario; a diferencia de la descarga de archivos, que requiere que el usuario descargue los 
archivos por completo para poder acceder a ellos.},tan sólo por nombrar algunas [12]\footnote{Página 1}. Esto 
conlleva a un importante crecimiento y evolución tanto de la metodología desarrollo como de las tecnologías 
utilizadas a la hora de crear una aplicación web.\\


El sitio Web es el medio más barato para darse a conocer rápidamente con un alcance mundial. 
Esto es extensible no sólo a empresas que comercializan productos y servicios, o a profesionales 
autónomos, sino que también a personas u organizaciones que actúan sin ánimo de lucro, que intentan
divulgar sus obras, inquietudes o ideas.\\

Al comienzo, los sitios Web ofrecían casi de forma exclusiva contenidos basados en texto 
y eran bastante estáticos; en la actualidad son sitios interactivos con abundancia de elementos multimedia.\\


Como se sabe, la evolución tecnológica se hace presente en todas y cada una de las áreas de investigación, 
tanto en las ciencias de la física y química que permiten la construcción de hardware más potente, como en 
los procesos y tecnologías de desarrollo de software, específicamente, y lo que es de interés para este documento, 
las tecnologías de Desarrollo Web.\\

%\subsection{Motivación}

\subsection{Definición del Problema}

Hoy en día las herramientas Web son fundamentales en todo ámbito de negocios, desde soluciones comerciales
como carritos de compra, catálogos empresariales, e incluso complejos sistemas \textit{Enterprise Resource
Planning} o ERP por sus siglas en inglés, que consisten en una arquitectura de software para empresas que 
facilita e integra la información entre las funciones de manufactura,logística, finanzas y recursos humanos 
de la empresa.\\

Como se mencionó anteriormente, a la evolución de los servicios ofrecidos por las diversas empresas qwue pertencen
al rubro, hay que sumar el continuo aporte y empoderamiento de los propios usuarios quienes crean y consumen
contenido a tasas cada vez más altas.\\

Como ya ha ocurrido con la evolución desde páginas web estáticas a sitios web dinámicos, o la aparación de tecnologías
asincrónicas, es natural esperar que esta evolución continue, después de todo la web evoluciona de acuerdo a las 
necesidades del usuario.\\

Es fundamental reconocer y probar las nuevas tecnologías que han ido apareciendo con el transcurrir de los años,
de ese modo, poder evaluar de forma concreta y objetiva, en que contexto son de utilidad.\\


\subsection{Objetivos}

\subsubsection{Objetivos Generales}
%separar en general y especifico (preguntarle al profe)!!!

El objetivo de esta memoria es realizar una evaluación de diversas tecnologías emergentes de desarrollo web
y compararlas versus una implementación de carácter tradicional bajo ciertos criterios. Para ello se deben cumplir 
los siguientes objetivos generales:

\begin{enumerate}
 \item Explicar el avance de las tecnologías de desarrollo web, y cómo han evolucionado desde páginas web
	estáticas hasta el concepto de aplicaciones web.
 \item Evidenciar empíricamente la diferencia de rendimiento entre tecnologías emergentes, utilizando como
	base de comparación, tecnología de carácter más tradicional.
% \item Obtener conclusiones y proponer sugerencias sobre la implementación de 
\end{enumerate}


\subsubsection{Objetivos Específicos}
\begin{enumerate}
 \item Investigar tecnologías existentes y emergentes para el desarrollo de aplicaciones web.
 \begin{enumerate}
   \item  Revisar evolución histórica de las tecnologías asociadas a desarrollo de aplicaciones Web.
   \item  Realizar un levantamiento del estado del arte.
 \end{enumerate}
 \item Investigar y evaluar cómo las tecnologías recientes pueden ayudar a desarrollar mejores aplicaciones 
	Web y  mejorar los procesos de desarrollo de software.
 \begin{enumerate}
   \item Definir criterios de clasificación general y selección de las tecnologías específicas a investigar
   \item Aplicar los criterios anteriores para seleccionar un conjunto de tecnologías a investigar en mayor profundidad.
   \item Evaluar en más detalle las tecnologías seleccionadas.
 \end{enumerate}
 \item Definir una aplicación de prueba y desarrollar un prototipo aplicando las tecnologías seleccionadas que permita 
	entender, integrar y evaluar estas tecnologías. Comparar con la aplicación de tecnologías más tradicionales.
 \begin{enumerate}
   \item Diseñar y desarrollar el prototipo aplicando diferentes tecnologías.
   \item Evaluar cada uno de los casos de aplicación
 \end{enumerate}
 \item Definir el tipo pruebas que midan rendimiento y como se aplicarán a los prototipos. 
 \item Hacer una evaluación global y obtener conclusiones.
\end{enumerate}


\subsection{Metodología de trabajo}
 
El desarrollo de este trabajo se divide en etapas, las cuales son:
\begin{enumerate}
 \item Definición de Objetivos y Estado del Arte: Esta etapa consiste en la definición de los Objetivos Generales 
 y Específicos de esta memoria, se investiga la evolución histórica de las diversas tecnologías de desarrollo web, 
 y se realiza un levantamiento del estado del arte algunas de las metodologías más importantes de desarrollo web.
 
 \item Selección de Tecnologías y Ambiente de Desarrollo: Esta etapa consiste en la definición y
 aplicación de criterios de selección que permitan filtrar y escoger de entre las tecnologías investigadas en la etapa 
 anterior. Por otra parte costa de la implementación de un ambiente de desarrollo necesario para realizar las diversas 
 pruebas requeridas en los objetivos. Dicha implementación es necesaria para cada tecnología escogida.
 
 \item Definición de pruebas y Experimentos: Esta etapa consiste en la definición las pruebas orientadas a medir el 
 rendimiento de los prototipos construidos. Una vez definidas, se procederá a correr los experimentos en los prototipos, 
 midiendo su desempeño.
 
 \item Analisis y Resultados: En esta etapa se analizarán y compararán los resultados obtenidos en la etapa anterior,
 explicando cada uno de ellos.
 
 \item Conclusiones: La última etapa consta de la obtención de conclusiones del trabajo realizado, tomando
 en cuenta todas las etapas anteriores.
\end{enumerate}

\subsection{Estructura del documento}

En el capítulo 1 del presente documento se realiza una descripción del problema, contextualizando 
al lector y motivandolo con las consecuencias de su resolución. Además se presentan los objetivos 
que busca cumplir esta memoria.\\

En el capítulo 2 se define que es una aplicación web, y se explican los procesos de desarrollo de 
software necesarios para poder construirla; comenzando por los procesos de desarrollo de software 
en general, los cuales son los precursores de los procesos de desarrollo web.\\
%llenar
%agregar kanban?

En el capítulo 3 se realiza una revisión sobre la evolución histórica de las tecnologías web.
Se explican las características y las tecnologías utilizadas más populares en su época.\\
%llenar
% agregar ror y django de todos modos

En el capítulo 4 se definen los criterios de selección bajo los cuales se escogerán las tecnologías a 
usar en el prototipo, ampliando la investigación de las tecnologías seleccionadas.\\
%php apache// express(js) nodejs // django y ror quedan en espera por ahora.

En el capítulo 5  se definen las pruebas y experimentos a realizar utilizando las tecnologías seleccionadas, 
se describe la preparación de los ambientes de desarrollo necesarios para poder realizar dichas pruebas y
se deja en manifiesto cualquier posible problema y su respectiva solución a la hora de preparar el
ambiente de desarrollo.\\

En el capítulo 6 se ejecutan las pruebas, se reunen, analizan y comparan los resultados obtenidos.\\

El capítulo 7  está dedicado a las conclusiones obtenidas en esta memoria. Se responderá a los
objetivos planteados en el capítulo 1.\\


\newpage

%Estado del arte

%	Hablar de las metodologías de desarrollo.
\chapter{Capítulo 2}
\section{Estado del Arte}

\subsection{Aplicaciones Web}

Una aplicación Web corresponde a un software basado en internet, donde una gran población de 
usuarios realizan peticione remotas a través de un navegador web y esperan rpuestas de un 
servidor web [2]\footnote{Página 90}. Corresponde a una aplicación que se codifica en un lenguaje 
soportado por los navegadores web. Es decir las aplicaciones web corresponden al modelo 
cliente-servidor.\\

Una aplicación web se distingue por el uso de \textit{hipermedia}, que conjuga tanto la tecnología 
hipertextual, como la multimedia. Si la multimedia proporciona una gran riqueza en los tipos de datos, 
el hipertexto aporta una estructura que permite que los datos puedan presentarse y explorarse siguiendo 
distintas secuencias, de acuerdo a las necesidades y preferencias del usuario.\\

En la actualidad existe una gran variedad de aplicaciones Web, que van desde páginas sólo de carácter 
informativo hasta aplicaciones complejas que ofrecen diversidad de servicios al usuario, ya sean interacción
con otros usuarios a través de juegos en linea, mensajería y \textit{streaming} entre otras.\\

Más allá del tipo de aplicación con el que se esté trabajando, es de suma importancia contar con un 
proceso de desarrollo de software, con el fin de estandarizar su proceso de creación. A continuación 
se exponen algunos de los procesos de desarrollo de software tanto general, como orientados al desarrollo 
de aplicaciones web\\


\subsection{Procesos de Desarrollo de Software de carácter general}

Desde el punto de vista de la ingeniería de software, es importante contar con los mecanismos
adecuados (métodos y tecnologías de desarrollo, hardware acorde a la aplicaciones a desarrollar, entre
otros) para que la creación de este tipo de aplicaciones satisfaga las necesidades tanto de los usuarios 
como de los clientes que contratan el desarrollo de este tipo de aplicaciones.
En cualquier caso, existen criterios universalmente aceptados acerca del desarrollo software. Por ejemplo, 
el modelo de proceso más adecuado para el desarrollo de software es un proceso iterativo e incremental, 
puesto que a diferencia de otros modelos de proceso, como el modelo en cascada, permite la 
obtención de diversas versiones del producto software antes de su entrega final, por ende permite su depuración 
y validación progresiva, lo que sin duda redundará en un software de mejor calidad. Además, con este tipo de 
proceso es posible añadir o modificar requisitos que no han sido detectados con anterioridad [12]\footnote{Página 2}.\\

Dentro de los procesos de desarrollo de software más importantes, son destacables Modelo de cascada y el Modelo 
iterativo incremental\\

Estas metodologías se conocen también como "tradicionales" e imponen una disciplina de trabajo sobre el proceso 
de desarrollo del software, con la meta de conseguir uno más eficiente y predecible. Para ello, se hace un especial 
hincapié en la planificación total de todo el trabajo a realizar y una vez que esta todo detallado, comienza el ciclo 
de desarrollo del producto software. Este planteamiento está basado en el resto de disciplinas de ingeniería, a 
pesar de que el software no pueda considerarse como la construcción de una obra clásica de ingeniería [12]\footnote{Página 2.}.\\

Una de sus principales deficiencias corresponde a que si existe un cambio, es posible que toda la planificación se 
venga abajo, es por ello que no son métodos adecuados cuando se trabaja en un entorno que pueda variar constantemente.\\

\subsection{Procesos de Desarrollo de Aplicaciones Web}

En los últimos años ha surgido un conjunto de métodos para desarrollar aplicaciones Web, los cuales presentan 
en forma explícita su modelo de proceso, es decir las actividades técnicas y gerenciales que son requeridas para el
desarrollo de la aplicación. \\

Los métodos que se presentarán se han agrupado de acuerdo a su modelo de proceso y al contexto particular donde 
pueden ser aplicados. Es necesario aclarar que, ninguno de ellos guía al grupo de desarrollo en la construcción de 
aplicaciones Web para múltiples contextos, debido a su entorno multivariable. Entre los métodos más conocidos se 
encuentran [2]\footnote{Páginas 90-92}:

\subsubsection{Métodos ágiles para el desarrollo de aplicaciones}
Estos procesos aportan como novedad, nuevos métodos de trabajo que apuestan por equilibrar la relación \textbf{proceso
-esfuerzo}. En otras palabras, ni se pierden en la excesiva burocracia de los métodos tradicionales, ni apuestan por una 
ausencia total de procesos.\\

Los procesos ágiles son una buena elección cuando se trabaja con requisitos desconocidos o variables. De no existir
requisitos estables, la posibilidad de utilizar un proceso "tradicional" no es recomendable. En estas situaciones, un 
proceso adaptativo se considera mucho más efectivo que un proceso predictivo. Los métodos ágiles para el desarrollo de 
software se caracterizan por poseer iteraciones cortas, pruebas continuas y frecuente replanificación basada en la 
realidad actual.\\

Dentro de las metodologías basadas en este enfoque destacan: Extreme Programming (XP), Open Source y Dynamic Systems 
Development Method (DSDM).

\subsubsection{Metodología de desarrollo Kanban}
\textit{agregar kanban development, sus principios derivdos de toyota y el managment}

\subsubsection{Métodos para el desarrollo de sistemas de información Web (SIW) }
Estos métodos se caracterizan por seguir una secuencia de fases y pasos requeridos para desarrollar SIW, y asegurar la 
calidad del mismo. Cubren todo el ciclo de desarrollo de un SIW, emplean técnicas de análisis, de  diseño orientado
a objetos y utilizan un enfoque iterativo. 
Son comparables al proceso de desarrollo iterativo incremental, pues integra los siguientes procesos [11]:
\begin{itemize}
 \item Análisis preliminar para sistemas de información Web
 \item Selección de lenguaje y desarrollo de la aplicación
 \item Implementación
 \item Mantenimiento
 \item Estándares y Documentación
\end{itemize}

Uno de los más conocidos corresponde a la Metodología para desarrollar Sistemas de Información Web (MIDAS).

\subsubsection{Métodos para el desarrollo de aplicaciones hipermedia}
La mayoría de estos métodos sólo cubren parcialmente el ciclo de desarrollo de las aplicaciones hipermedia, dando
especial importancia al diseño. Por lo general utilizan dos técnicas en cualquier diseño de aplicaciones hipermedia:
Modelo Entidad-Relación y técnicas de Orientación a Objetos. 

Los más difundidos son: Aplicando modelos de proceso
de software al desarrollo de aplicaciones hipermedia, The Object-Oriented Hypermedia Design Model (OOHDM)
Relationship Management Methodology (RMM), Hypermedia Flexible Process Modeling (HFPM) y el Enfoque de Ingeniería 
de Lowe-Hall’s.

\subsubsection{Métodos para el desarrollo de sitios Web de aprendizaje (e-learning)}
El objetivo de estos métodos es ayudar a los diseñadores de los cursos y profesores a desarrollar sitios Web de aprendizaje
entendibles, por lo tanto se incorpora gran variedad de componentes organizacionales, administrativos, didacticos y 
tecnológicos, pues estos métodos consideran el elemento humano (alumnos, profesores, ayudantes, administradores), 
los recursos de aprendizaje basados en Web, otros recursos de aprendizaje (textos o guías) y la infraestructura tecnológica 
necesaria para desarrollar el proceso de aprendizaje. Cabe destacar que hacen énfasis en la reutilización de componentes 
para reducir el tiempo y costo de desarrollo. 

Los más conocidos son: Desarrollo de sitios Web instruccionales – Un Enfoque de Ingeniería de Software, Simple Web Method (SWM) 
y el Modelling Web-Based Instructional Systems.


\subsubsection{Métodos para el desarrollo de aplicaciones de comercio electrónico: (e-commerce)}
Estos métodos están basados en el reciclado e integración de componentes de software con el fin de lograr funcionalidades 
empresariales para aplicaciones de e-commerce (negociación, mediación; “workflow” interempresarial y notificaciones de
eventos). 

Uno de los más difundidos es el Marco de Referencia Basado en componentes para e-commerce.


\newpage

%	Hablar de la evolución de las aplicaciones web (paper de avance)
\chapter{Capítulo 3}


\section{Tecnologías utilizadas en el Desarrollo de aplicaciones Web}

\subsection{Un poco de historia}
Antiguamente, la creación de un sitió web se limitaba sólo  a escribir cada 
página directamente con mediante código HTML. Esta tarea es factible solamente
en sitios cuyo contenido es limitado y sus actualizaciones son casi nulas,
características propias de la \textit{Web 1.0} es decir, las típicas  que ostentaban 
los sitios durante la primera mitad de los años 90.\\

Posteriormente aparecen los lenguajes de desarrollo Web intentan facilitar las 
tareas de los creadores de aplicaciones, de manera que se automaticen los procesos, 
y permitan entrar al juego a los usuarios, pasivos hasta ese momento, es decir
se crea la web 2.0.\\

Actualmente, y gracias al nacimiento de las redes sociales, el papel que juegan los 
usuarios en la web es cada vez más importante; tanto así que su uso ha permitido realizar 
cambios tan drámaticos en la sociedad entre los que destacan por ejemplo, comunicarse 
con personas en casi cualquier parte del mundo o el auge de levantamientos populares en 
paises como Egipto. 

Por lo tanto, mientras que con HTML sólo es posible crear sitios Web estáticos, 
utilizando lenguajes de desarrollo Web es posible crear sitios Web dinámicos. Se 
conoce con el nombre de sitio Web dinámico a aquel cuyo contenido se genera a partir 
de lo que un usuario introduce en un web o formulario. 


\subsection{Web 1.0}

\subsubsection{Características}

La Web 1.0 o "web estática" (1991-2003) es la forma más básica que existe, con navegadores 
de sólo texto bastante rápidos. La aparición de HTML hizo que las páginas web fuesen más 
agradables a la vista. Paralelamente aparecen los primeros navegadores visuales tales como 
Internet Explorer y Netscape [3]. Su principal característica es que es de sólo lectura, es 
decir que para el usuario no es posible interactuar con el contenido de la página estando 
totalmente limitado a lo que el Desarrollador sube a ésta.

\subsubsection{Tecnologías de desarrollo}

Si bien, la tecnología predominante en la web 1.0, es el código HTML, es necesario poder 
realizar las transferencias necesarias de información entre cliente y servidor, información que 
principalmente correpsondía a hipertexto. Es por ello que se creó un protocolo, es decir una serie 
de reglas utilizadas por los computadores para poder realizar transferencias de archivos de datos,
sonido o imagen. Dicho protocolo corresponde a HTTP o acrónimo de \textit{HyperText Transfer Protocol}.\\

Desde 1990, el protocolo HTTP  es el más utilizado en Internet. Si bien su versión 0.9 sólo tenía 
la finalidad de transferir los datos a través de Internet, su versión 1.0 permite la transferencia de 
mensajes con encabezados que describen el contenido de los mensajes mediante la codificación MIME.\\

El propósito del protocolo HTTP es permitir la transferencia de archivos principalmente, en formato 
HTML. entre un navegador y un servidor web,  mediante una cadena de caracteres conocida como dirección 
URL.\\ 

HTML corresponde al acrónimo de \textit{HyperText Markup Language} y se trata de un conjunto de 
etiquetas que sirven para definir el texto y otros elementos que compondrán una página web. Básicamente 
este lenguaje indica a los navegadores cómo deben mostrar el contenido de una página web.\\

HTML se creó con el objetivo de divulgar información, principalmente texto y posteriormente texto 
con imágenes. Creado en 1986 por el físico nuclear Tim Berners-Lee [9], no se pensó que llegaría  a 
ser utilizado para crear sitios de consulta con carácter multimedia. Sin embargo, pese a esta deficiente 
planificación, se han ido incorporando modificaciones con el tiempo, estos son los estándares del 
HTML\footnote{El estándar actual corresponde a HTML5} [8].\\

Desde el punto de vista del webmaster, la tarea de mantención de la página es relativamente simple, 
considerando la base de la web 1.0. Sin embargo, se convierte en una tarea titánica en aquellos sitios con 
muchos contenidos y que incorporan frecuentemente novedades. Por ejemplo, si se quieren realizar en HTML 
cambios sobre algún elemento común a todas las páginas del sitio, se deben aplicar en todas las páginas,
una por una, con lo que se  convierte en un trabajo muy tedioso. Por lo cual nace la necesidad de 
integrar al usuario a la faceta de creación y mantención de contenidos.\\

Como ya se ha mencionado, la tecnología preponderante es HTML, introducción de formularios y CGI. 
No obstante, también en esta etapa de desarrollo surgen lenguages de scripting para la Web, tales 
como: PHP, ASP y JSP.\\

\begin{itemize}
 \item CGI: denominado CGI \footnote{Common Gateway Interface por sus siglas en inglés, o 
	    Interfaz de Puerta de Enlace Común.}. Es posible calsificarlo como en el límite, pues 
	    corresponde a la primer intento de realizar el salto de contenidos estáticos a contenidos 
	    dinámicos. En las aplicaciones CGI, el servidor web pasa las solicitudes del cliente a un 
	    programa externo.  La salida de dicho programa es enviada al cliente en lugar del archivo 
	    HTML tradicional. El cliente se encarga de interpretar esta salida.
	    
 \item PHP: Una forma de generar el contenido dinámico, corresponde a que sea el servidor quien ejecute 
	    las secuencias de comandos para generar la página HTML. PHP \footnote{Hypertext Pre-processor 
	    por sus siglas en inglés.}  tiene la ventaja de ser gratuito y versátil, pues es soportado por 
	    la mayoría de los sistemas operativos y servidores; además de contar con múltiples herramientas
	    de desarrollo como frameworks\footnote{Framework es un concepto sumamente genérico, se refiere 
	    a “ambiente de trabajo, y ejecución”.En general los framework son soluciones completas que 
	    contemplan herramientas de apoyo a la construcción (ambiente de trabajo o desarrollo) y motores 
	    de ejecución (ambiente de ejecución).} donde destacan PHPCake o Symfony. Para utilizar PHP, el 
	    servidor Web debe entenderlo. Por lo general, las páginas Web que contienen comandos PHP utilizan 
	    la extensión “.php” en lugar de “.html”. De todos modos, el cliente nunca ve el código PHP, sino 
	    los resultados que produce en código HTML.

 \item ASP.NET: Una alternativa es la que ofrece Microsoft para generar sitios Web dinámicos, conjuntamente
		con su software servidor IIS (Internet Information Server). Se trata de ASP \footnote{Active 
		Server Pages}, que desde su primera versión ha evolucionado hasta denominarse ASP.NET y estar 
		dentro de la plataforma “.NET”. Una de las principales ventajas de ASP.NET es la gran cantidad 
		de lenguajes que soporta. ASP.NET constituye un entorno abierto en el que se puede combinar 
		código HTML, scripts y componentes ActiveX del servidor para crear soluciones dinámicas y de 
		calidad para la Web. Las páginas que utilizan esta tecnología tienen la extensión “.asp”.
		
\end{itemize}

\subsection{Web 2.0}

\subsubsection{Características}

El término Web 2.0 está asociado a aplicaciones web que están desarrolladas para compartir información, 
pues su diseño está centrado en el usuario. Un sitio Web 2.0 está pensado para que los usuarios puedan 
interactuar y colaborar entre sí [3], tomando el rol de creadores de contenido generado por ellos mismos 
en una comunidad virtual, lo cual es diametralmente opuesto al concepto de pasividad del usuario, algo 
predominante en la web 1.0.\\

Por lo tanto, la Web 2.0 es una evolución del viejo concepto de cómo se usa la web, de manera unidireccional, 
como consumidores pasivos. El término, acuñado por Tim O'Reilly\footnote{En [6], realiza interesantes comparaciones 
entre los elementos de la web 1.0 y la web 2.0} en una conferencia del renacimiento y la evolución de la web, 
designa una nueva forma de servicios web basados en la participación de los usuarios, quienes conforman el 
motor básico del sistema de información.\\

Debido al aumento en la participación de los usuarios, nacen premisas como: “todos tienen algo que decir 
y todos pueden hacerlo” (Orihuela, 2006). El volumen de datos generados es tal, que se necesitan sistemas 
de filtrado, clasificación y organización de la información; sistemas que además, deben estar basados en la 
arquitectura de la participación y la inteligencia colectiva [5].\\


El hecho de que la Web 2.0 sea cualitativamente diferente de las tecnologías web revisadas en el 
apartado anteriores ha sido cuestionado por el creador de la World Wide Web Tim Berners-Lee, quien 
calificó en su tiempo al término como "tan sólo una jerga", pués tenía la intención de que la Web 
incorporase estos valores en el primer lugar. 


\subsubsection{Tecnologías de desarrollo}

Ante la necesidad de empoderar a los usuarios, fueron surgiendo varios lenguajes de programación y 
tecnologías orientadas al desarrollo web necesarios para lograr la creación de sitios dinámicos.\\

Un sitio web dinámico se puede generar a través de secuencias de comandos en un servidor web; cuando 
el cliente web recibe la respuesta, la trata como una pagina HTML y la despliega. Un ejemplo de esto 
es cuando un usuario rellena los campos de un formulario y realiza el envío de la información, al 
momento de llegar al servidor dicha información se entrega a un programa o secuencia de comandos
para que sea procesada, que por lo general corresponde a una interacción con una base de datos y 
la generación de una página HTML con información personalizada, la cual es reenviada al cliente.\\

Por lo general, los sitios web dinámicos, estan compuestos por la combinación:
\begin{itemize}
 \item Plataforma del servidor web: Apache, Tomcat, entre otros.
 \item Gestor de base de datos, que por lo general es de carácter relacional: Oracle, PostgreSQL,
	Microsoft SQL Server, MySQL, entre otros.
 \item Lenguajes y tecnologías de programación web: Perl, PHP, JSP, JavaScript, entre otros.
\end{itemize}

Dependiendo de las necesidades, se define la combinación adecuada. 

%Se revisarán algunos de los Lenguajes y tecnologías para el desarrollo web, relegando tanto la plataforma 
%del servidor como el gestor de base de datos, pues se escapan del alcance de la investigación.

Algunas de las tecnologías de la llamada web 2.0 son:

\begin{itemize}

 \item JavaScript: A pesar de la gran potencia de las tecnologías anteriores, ninguna de ellas 
		   puede responder, por ejemplo, a los movimientos del ratón o interactuar de manera 
		   directa con los usuarios. Para lograr esto, es necesario tener secuencias de comandos 
		   embebidas en las páginas HTML, pero que a diferencia de, por ejemplo PHP,  se ejecuten en 
		   la máquina cliente y no en el servidor. JavaScript es un lenguaje de scripts interpretado 
		   que se integra directamente en páginas HTML (a veces por modularidad se separa en 
		   ficheros con extensión “.js”) y es interpretado, en su totalidad, por el cliente Web 
		   en tiempo de ejecución, sirviendo así para todos los sistemas operativos.
		   
 \item AJAX: O \textit{Asynchronous JavaScript And XML} es un término que engloba la utilización 
	     de varias tecnologías, para crear aplicaciones Web dinámicas que se ejecutan en el cliente. 
	     Entre estas tecnologías destacan JavaScript, XML, XHTML, HTML y CSS. Al realizar cambios 
	     sobre la misma página sin necesidad de recargarla, se consigue un notable aumento de 
	     interactividad y velocidad.
	     
 %\item Blogs y Redes sociales:
 \end{itemize}


 
\subsection{Actualidad}

\subsection{Características}
 
 La Web actual se caracteriza por premiar la creatividad de los usuarios, fomentar su participación 
y transmisión del conocimiento entre pares potenciando el sentido de comunidad. En esta nueva época 
la Web ya no es meramente informativa, sino que el usuario toma rol consumidor, intercambiador y creador 
de contenidos. Los sitios web se convierten en fuentes de contenido y expresión para los usuarios. Es 
aquí donde de fundan los actuales cimientos de la Web: redes sociales y sabiduría de las multitudes, tan sólo por 
nombrar algunas.

\subsection{Tecnologías}
Dentro de las tecnología utilizadas en esta web se encuentran:

\begin{itemize}
 
 \item Redes Sociales: Las Redes Sociales son páginas web que permiten a las personas conectarse con 
  sus amigos e incluso realizar nuevas amistades, a fin de compartir contenidos, interactuar y crear 
  comunidades sobre intereses similares. Gracias a la alta tasa de conexión a internet y su posterior 
  desarrollo como medio de comunicación de masas, ha posibilitado que las redes sociales puedan existir, 
  además de en el espacio físico, en el espacio virtual. Esto facilita su extensión a lo largo de 
  diferentes regiones y del mundo, superando para siempre el factor geográfico que muchas veces las limitaba.
  Cabe destacar que han sido (y son) un factor de cambio clave dentro de la sociedad actual, en ambitos tan
  distintos, que van desde llevar relaciones amorosas al plano virtual hasta ser el medio de comunicación 
  primario a la hora de convocar manifestaciones.
 
 \item Mushup (Web Híbrida): Corresponde a aplicaciones webs que contiene a su vez aplicaciones webs. 
  Consta por lo general, en el uso de las API de variadas compañías. Una situación recurrente es el uso 
  de Google Maps,  que corresponde a una aplicación web, dentro de páginas que se dedican al seguimiento 
  de, por ejemplo, vehículos a través de un sistema GPS; lo que corresponde a otra aplicación web.
 
 \item QOOXDOO \footnote{http://qooxdoo.org/}: Qooxdoo es un framework universal de JavaScript que permite 
  crear aplicaciones para una amplia gama de plataformas. Qooxdoo aprovecha las tecnologías web más modernas, 
  como HTML5 y CSS3. Es de código abierto, está totalmente basado en clases y trata de aprovechar las características 
  de orientación a objetos de JavaScript. Se basa completamente en los espacios de nombres y no se extiende tipos 
  nativos de JavaScript para permitir una fácil integración con otras bibliotecas y código de usuario existente.[7]
  Una aplicación típica de qooxdoo se crea mediante el aprovechamiento de las herramientas de desarrollo integrado 
  y el modelo de  programación del lado del cliente basada en orientación a objetos de JavaScript.
  Algunas de sus características son:
   \begin{enumerate}
    \item Qooxdoo soporta una amplia gama de entornos de JavaScript, tales como navegadores convencionales
      \footnote{Internet Explorer, Firefox, Opera, Safari, Chrome} y móviles \footnote{iOS, Android}
    \item No necesita plugins (ActiveX, Flash, Silverlight)
    \item Mantiene objetos nativos de JavaScript, con el fin de permitir una fácil integración con bibliotecas 
	  y código personalizado. 
    \item Al estar bajo el paradigma de la orientación a objetos, está basado en clases (en su totalidad). 
	  Además soporta clases abstractas.
    \item Cuenta con soporte completo para programación basada en eventos
    \item El desarrollo de aplicaciones qooxdoo es totalmente compatible con todas las plataformas, como Windows, 
	  todos los sistemas Unix (Linux), Mac OS X.
    \item Cuenta con muchas aplicaciones de muestra y ejemplos.
   \end{enumerate}
   
  \item Node.js\footnote{http://nodejs.org/}: Es un entorno de programación basado en el lenguaje de 
    programación Javascript, con I/O de datos y una arquitectura orientada a eventos. Fue creado con el 
    enfoque de ser útil en la creación de programas de red altamente escalables, como por ejemplo, servidores 
    web. Las primeras encarnaciones de JavaScript vivían en los browsers, es decir en el frontend. Sin embargo, 
    lo anterior es solo un contexto, pues JavaScript es un lenguaje "completo"; es decir, se puede  usar en muchos 
    contextos y alcanzar con éste, todo lo que se puede alcanzar con cualquier otro lenguaje "completo".
    Por ello Node.js realmente es sólo otro contexto: permite correr código JavaScript en el backend, fuera del 
    browser. Para ejecutar el código JavaScript que se pretende correr en el backend, debe ser interpretado y 
    ejecutado. Node.js se encarga de esta tarea haciendo uso de la Maquina Virtual V8 de Google; que por lo demás 
    es el mismo entorno de ejecución para JavaScript que Google Chrome utiliza. Además, Node.js viene con muchos 
    módulos útiles \footnote{Usando Node Package Module o NPM por sus siglas en inglés.}, de manera que no hay 
    que escribir todo de cero.

 \item V8 Engine \footnote{http://code.google.com/p/v8/}: Es un motor de JS desarrollado por Google. La 
 ejecución de programas JS se realiza compilando el código, aumentando el desempeño respecto a Java ejecutado en 
 lenguaje  interpretado Bytecode. Algunas de sus características son:
 \begin{enumerate}
  \item Está escrito en C++ y es usado en Google Chrome.
  \item Está integrado en el navegador de internet del sistema operativo Android, al
	menos desde su versión Froyo.+
  \item Corre en Windows (desde la versión XP), Mac OS X 10.5 (Leopard) y Linux en procesadores 
	IA-32 y ARM.
  \item V8 puede funcionar de manera individual (standalone) o incorporada a cualquier aplicación 
	C++.
 \end{enumerate}

 \item MongoDB \footnote{http://www.mongodb.org/}: MongoDB es un sistema de base de datos
  multiplataforma orientado a documentos, de esquema libre. Al estar escrito en C++ le confiere cierta 
  cercanía a los recursos de hardware de la máquina, de modo que es bastante rápido a la hora de 
  ejecutar sus tareas. Al ser NoSQL, hay que olvidarse de las tablas y las relaciones entre ellas.
  En MongoDB, cada registro o conjunto de datos se denomina documento. Los documentos se pueden 
  agrupar en colecciones, las cuales se  podría decir que son el equivalente a las tablas en una base 
  de datos relacional (sólo que las colecciones pueden almacenar documentos con muy diferentes formatos, 
  en lugar de estar sometidos a un esquema fijo). Se pueden crear índices para algunos atributos de los 
  documentos, de modo que MongoDB mantendrá una estructura interna eficiente para el acceso a la información 
  por los contenidos de estos atributos. [13] Se abandona el enfoque relacional por bases de datos mas 
  orientadas a objetos y de esta manera es como se procesa la información.
 
 \item Kendo \footnote{http://www.kendoui.com/}: Bajo el lema “El arte del desarrollo web” Kendo UI 
  ofrece un completo abanico de posibilidades, siendo un Framework para la creación dinámica de mo-dernas 
  interfaces se vale de las virtudes de HTML5, CSS3 y jQuery para generar potentes elementos perfectamente 
  compatibles con los navegadores más modernos como también para los dispositivos móviles más utilizados 
  en la actualidad [15]. La compatibilidad de Kendo UI es uno de sus puntos fuertes, ya sea con modernos 
  navegadores o sistemas operativos móviles:
  \begin{enumerate}
   \item Internet Explorer 7+
   \item Firefox 3+
   \item Safari 4+
   \item Chrome
   \item Opera 10+
   \item Android 2.0+
   \item iOS 3.0+
   \item BlackBerry OS 6.0+
   \item webOS 2.2+
  \end{enumerate}
 
 \item Django \footnote{https://www.djangoproject.com/} (Python): Django es un framework web basado en Python,
  cuya meta es un "desarrollo rápido, limpio y con un diseño pragmático". De acuerdo a la información
  otorgada por su sitio web, cuenta con un Object-Relational Mapping \footnote{ORM, por sus siglas en inglés},
  que si bien, por defecto está configurada para operar una base de datos SQLite, puede modificarse para 
  trabajar con una base de datos MySQL u otras. Se utiliza tanto para levantar un servidor web, como para 
  crear el código fuente del sitio en construcción. Además cuenta con una gran comunidad de programadores que
  aportan soluciones a los diversos bugs que aparecen.

 \item Ruby on Rails \footnote{http://rubyonrails.org/} (Ruby): 
\end{itemize}

 
\subsection{Evolución}

\subsubsection{Características}
Actualmente se está viviendo otra revo-lución, términos como web semántica y web 3.0 o computación 
ubicua con web 4.0 son cada vez más comunes.

Utilizar números para marcar generaciones sucesivas de la Web parece una buena idea cuando se 
comprueba el éxito que tuvo la denominación 2.0. ¿Cuáles serían los rasgos de esta futura Web? Aquí 
se entra en un terreno difícil, puesto que no se trata de algo existente en laactualidad, sino de una 
especulación acerca de cómo se va encaminado la Web ahora y en el futuro cercano. Una forma de solucionar el 
problema es lo que hacen algunos analistas y que consiste en identificar Web 3.0 con Web Semántica. Otros 
analistas los ven de forma separada, donde esta nueva web tiene las siguientes características:

\begin{itemize}
 \item Computación en la nube y Bases de datos no relacionales
 \item Agentes de usuario (como en la Web Semántica)
 \item Anchura de banda
 \item Mayor acceso a internet
\end{itemize}

Si bien los dos últimos puntos son de carácter técnico, sin duda están teniendo repercusiones sociales. 
A mayor ancho de banda se facilita la ejecución de aplicaciones multimedia, mientras que un mayor acceso 
a internet no implica sólo que hay mayor cantidad de conexiones, sino que es posible conectarse desde toda 
clase de aparatos electrónicos, como celulares, tablets e incluso vehículos. Esto a su vez incentiva la 
creación e investigación de nuevas tecnologías a la hora de desarrollar aplicaciones web, ya sea para estandarizar
aspectos entre dispositivos, plataformas, sistemas operativos; o bien para personalizar cada aplicación con el 
fin de sacarle el máximo provecho en un dispositivo en particular.\\

De todos modos y en definitiva, de eso trata la Web 3.0, de páginas capaces de comunicarse con otras páginas 
mediante procesamiento de lenguaje natural y, es justo aquí cuando cobra sentido el nexo entre la Web Semántica 
y la Web 3.0. Ésta es la principal interpretación que se hace de éste término. 

\subsubsection{Tecnologías de desarrollo}

Algunas de las tecnologías de reciente aparición (y no tan recientes), de carácter libre y que se están aplicando en el 
área del desarrollo web, son:

\begin{itemize}
 \item Web Semantíca: A fines de la década de 1990, comenzó a idearse un nuevo cambio en la Web. Era un cambio a 
  la vez más complejo y más profundo que el que ha representado la Web 2.0. Se trataba del proyecto de la Web
  Semántica. A grandes rasgos, el objetivo fundacional de la Web Semántica consistió en desarrollar una serie de 
  tecnologías que permitieran a las computadoras, a través del uso de agentes de usuarios parecidos a los 
  navegadores actuales, no solo “entender” el contenido de las páginas web, sino además efectuar razonamientos 
  sobre el mismo. La idea era conseguir que el enorme potencial de conocimiento encerrado en documentos como 
  las páginas web pudiera ser interpretado por las computadoras de una forma parecida a como lo haría 
  un ser humano.[14]\\
  
 \item Computación Ubicua: Corresponde al acceso a gran cantidad de información y a su procesamiento de 
 forma independiente a  la ubicación de los usuarios. Esto implica la existencia de una gran cantidad de 
 elementos de computación disponibles en un determinado  entorno físico y constituidos en redes. Los elementos 
 están empotrados o embebidos en enseres, mobiliario y electrodomésticos comunes y  comunicados en red [16]. 
 Se suele asociar a términos como Computación Pervasiva e Inteligencia Ambiental

\end{itemize}
\newpage
%	Investigación que permite la elección de las tecnologías (hasta ahora php(apache) ror/django(node))

%agregar la arquitectura a utilizar
\section{Arquitectura}

%explicar en parrafo pequeño que es una arquitectura de software
%\textit{agregar parrafo introductorio}\\
Una arquitectura de software corresponde a un patrón de referencia que brinda
un marco necesario para guiar la construcción de software y establece la estrucutura de 
funcionamiento  e interacción entre sus diversas partes [20]. Este concepto 
"se refiere a la estructuración del sistema que, idealamente, se crea en las etapas 
tempranas de desarrollo"[21].\\ 

Al crear un software (independientemente de la metodología que se utilice), es necesario
cumplir una serie de pasos  que preceden a su construcción:
\begin{itemize}
 \item Requerimientos: Se enfoca a la captura y priorización de necesidades a satisfacer,
  ya sean de calidad, rendimiento y/o reestrictivas. Estos requerimientos son preponderantes
  e influencia la desición acerca de que arquitectura a utilizar.
 \item Diseño: Corresponde a la fase central en relación con la arquitectura. Debe satisfacer todos
  los requerimientos y no solo utilizar tecnologías de moda.
 \item Documentación: La documentación un factor crucial para comunicar un diseño de forma exitosa.
  Generalemente se utiliza la representación de varias de sus estructuras mediante el uso vistas. 
  Una vista generalmente contiene un diagrama, además de información adicional, que apoya en la 
  comprensión de dicho diagrama.
 \item Evaluación: Es de suma importancia evaluar el diseño una vez que este ha sido documentado 
  con el fin de identificar posibles problemas y riesgos. Evaluar (y validar) el diseño antes
  de codificar, disminuye el costo de corrección de errores.
\end{itemize}


Los grandes objetivos de una arquitectura de software son :
\begin{enumerate}
 \item Servir como guía durante el proceso de desarrollo. 
 \item Definir y satisfacer los atributos de calidad. 
\end{enumerate}
  

%explicar pq es necesario definir una arquitectura
Como dice Danny Thorpe:
\begin{center}
 \textit{"Programar sin una arquitectura en mente, es como explorar una gruta sólo con una
 linterna: no sabes dónde estás, dónde has estado ni hacia dónde vas"} [18]
\end{center}
La importancia de contar con una buena arquitectura que se adecue a situaciones reales, donde 
las diversas herramientas serán probadas. Es por ello que la arquitectura debe ser la respuesta 
ante un problema, no una imposición. Por otra parte, el desarrollo de software dejó de ser, hace mucho
tiempo, el trabajo de una o dos personas, pasando a ser equipos; por tanto es necesario facilitar la
comunicación entre sus integrantes.\\

A la hora de diseñar un software, hay que tener en cuenta que la arquitectura a utilizar se 
describe utilizando varios tipos de modelos[22]:
\begin{itemize}
 \item Estructurales: Se centran en la estructura coherente del sistema completo, en lugar de centrarse 
  en su composición. Representan todo como una colección organizada de componentes. 
 \item Frameworks: Identifican patrones de diseño repetibles, los cuales se pueden encontrar en aplicaciones
  similares.
 \item Dinámicos: Se centran en los aspectos del comportamiento dinámico de la arquitectura,  indicando
  como la estructura o la configuración del sistema puede cambiar en función de eventos externos.
 \item De Procesos: Se enfocan en el diseño de los procesos del negocio que el sistema debe soportar.
\end{itemize}

Dependiendo de la situación, se debe tomar la desición de que tipo de arquitectura utilizar.\\



Algunos elementos de una arquitectura son:
\begin{itemize}
 \item Cliente y Servidor
 \item Base de Datos
 \item Componentes
 \item Sistemas de Nivel Jerárquico
\end{itemize}

De entre las diversas arquitecturas, algunas de las más relevantes son:
\begin{itemize}
 \item Cliente-Servidor
 \item Orientado a Servicios
 \item Modelo Vista Controlador
 \item Modelo de N-Capas
\end{itemize}
%\textit{Pregntar al profe si es necesario explicar cada uno} basta con explicar y adjuntar diagrama




Considerando el hecho de que uno de los objetivos de esta memoria es, realizar pruebas y obtener
resultados concretos acerca de diversas herramientas de desarrollo web, es necesario enmarcar
dichas pruebas bajo un estándar que asegure una base pareja para todas ellas.\\

Es por dicha razón que se utilizará una arquitectura basada en 3 capas, de tipo \textbf{estructural} y 
\textbf{framework}.\\

\subsection{Arquitectura de 3 capas}
Una arquitectura bastante popular y de carácter relativamente simple, es la de 3 capas, cuyo
objetivo primordial es la separación de la aplicación en [23]:

\textit{Crear y agregar una imagen (con su respectiva explicación)}\\

\begin{enumerate}
 \item Una capa de presentación: Corresponde a la capa que ve el usuario, es decir, presenta 
  el sistema al usuario. Se responsabiliza de que se le comunique información al usuario por 
  parte del sistema y viceversa. La tarea de capturar información sobre el usuario se realiza,
  por lo general a través de formularios. Esta capa se comunica exclusiamente con la capa lógica.
  
 \item Una capa lógica/de negocios: Es la encargada de recibir, procesar y responder las peticiones
  que recibe, tanto del usuario a través de la capa de presentación, como las de la Base de Datos mediante
  la capa de datos. En otras palabras es una capa intermedia.
 
 \item Una capa de datos: Es la encargada acceder a los datos y está formada por un (o más) gestor de 
  base de datos. Dicho gestor recibe las peticiones de almacenamiento, recuperación, actualización y/o 
  eliminación de datos; desde la capa lógica. 
\end{enumerate}

Basicamente el objetivo principal es separar las distintas lógicas de la aplicación en niveles y que
éstos posean estructuras bien planteadas.\\

Para el caso de esta memoria, las pruebas se centrarán en la capa lógica y de presentación. Dejando
la capa de datos como una constante.
%Uso de arquitectura 3-tier
  %explicar de que se trata y pq se usa
  %hacer el diagrama 

%mapear las capas a la aplicaciòn



\newpage

% Criterio de seleccion
\chapter{Capítulo 4}

\section{Criterio de Selección}

El criterio de selección de tecnologías a utilizar, es un punto de importancia no menor en 
el desarrollo de esta memoria, pués consta de la combinación de una serie de herramientas que, 
dependiendo de cuáles y cómo se combinen, puede repercutir de manera positiva o negativa, el 
rendimiento del sistema y por lo tanto los resultados de las pruebas a realizar.\\

Para simplificar la tarea, se ha divido la selección en 2 items:
\begin{enumerate}
 \item Tecnologías tradicionales.
 \item Tecnologías emergentes.
\end{enumerate}

Dentro de cada item se utiliza la siguiente subdivisión:
\begin{itemize}
 \item Gestor de Base de Datos.
 \item Servidor Web.
 \item Lenguaje de programación Web y uso de Framework.
\end{itemize}



\subsection{Tecnologías Tradicionales}

Para la combinación de tecnologías tradicionales se utilizará:
\begin{itemize}
 \item Gestor de Base de Datos SQL
 \item Servidor Web Apache
 \item Lenguaje de Programación PHP y Framework Symfony \footnote{http://symfony.com/}
\end{itemize}

La razón de esta elección es debido a la gran popularidad, conviertiendose en practicamente una
combinación estándar durante la web 2.0. \\

La razón de utilizar una combinación de carácter más tradicional, es la de tener una base 
de comparación respecto a las tecnologías relativamente recientes, y de acuerdo a los
posibles resultados de las pruebas a realizar, determinar si las diferencias en cuanto a 
rendimiento son lo suficientemente altas (o bajas), para justificar tener (o no tener) que 
migrar sistemas y aplicaciones completas.\\
%buscar algún gráfico
%Explicar el porque

\subsection{Tecnologías Emergentes}

Para la combinación de tecnologías emergentes se utilizará:

\begin{itemize}
 \item Gestor de Base de Datos SQL
 \item Servidor Web Nodejs
 \item Lenguaje de Programación Javascript y Framework Express \footnote{http://expressjs.com/}
\end{itemize}

%Explicar el porque de la elección.
La razón para elegir esta combinación, radica principalmente en el hecho de poder utilizar
un lenguaje único, tanto para el servidor web como para desarrollar la aplicación web en sí.\\



%agregar al menos una combinacion más, con django y/o RoR


%porque nodejs
\subsubsection{¿Por qué Nodejs?}

Nodejs consiste en un framework de Javascript basado en el motor V8.\\

La meta número uno declarada de Node es \textit{proporcionar una manera fácil para construir 
programas de red escalables}. ¿Cuál es el problema con los programas de servidor actuales? 
En lenguajes como Java y PHP, cada conexión genera un nuevo \textit{thread} que potencialmente viene 
acompañado de 2 MB de memoria. En un sistema que tiene 8 GB de RAM, esto da un número máximo teórico 
de conexiones concurrentes de cerca de 4.000 usuarios.\\

Bajo el supuesto de que el impacto de una empresa ficticia es tal que crece la base de clientes, se desea 
que la aplicación soporte más y más usuarios, por lo tanto es necesario agregar más y más servidores.
Lo que representa una suma en cuanto a costos de servidor, tráfico, y por lo tanto potenciales
problemas técnicos: un usuario puede estar usando diferentes servidores para cada solicitud, así que cualquier 
recurso compartido debe almacenarse en todos los servidores. Por todas estas razones, el cuello de botella en 
toda la arquitectura de aplicación Web (incluyendo el rendimiento del tráfico, la velocidad de procesador y la 
velocidad de memoria) corresponde al número máximo de conexiones concurrentes que puede manejar un servidor.\\

Node resuelve este problema cambiando la forma en que se realiza una conexión con el servidor. En lugar de 
generar un nuevo hilo de SO\footnote{Sistema Operativo} para cada conexión y de asignarle la memoria necesaria, 
cada conexión dispara una ejecución de evento dentro del proceso del motor de Node.\\

Corre en la máquina virtual de javascript de google conocida como v8 y se basa en el paradigma
de programación dirigido a eventos de entrada y salida asincrónica.\\
%agregar imagen
%\includegraphics[]{×}

Una importante característica es que puede levantar servidores web en casi cualquier puerto que se 
le indique, además de posibilitar la comunicación única bidirideccional entre el cliente y el servidor[4],
mediante el uso de sockets o bien llamadas de tipo comet \footnote{Una llamada de tipo comet 
permite a un servidor web enviar información al navegador web si que este se lo pida explicitamente.}.

%http://book.mixu.net/ch13.html

De la misma forma que ocurre al trabajar con Apache, es posible expandir la funcionalidad de Node 
instalando módulos. Los módulos que se pueden utilizar con Node mejoran en gran medida el producto.
Así de importantes se han tornado los módulos, hasta el punto de convertirse en parte esencial del 
producto completo [10]. Los módulos se instalan usando Node Package Manager.\\

En lo que a plataformas se refiere y de acuerdo a la página oficial [17], están habilitadas las 
descargas para:
\begin{itemize}
 \item Windows
 \item Mac OS X
 \item Linux
 \item SunOS
\end{itemize}

\subsubsection{¿Por qué SQL en lugar de NoSQL?}
%Parrafo que explique el uso de sql
No se utiliza un gestor de Base de Datos NoSQL, ya que el objetivo es medir el comportamiento de
la combinación de servidor web y lenguaje de programación, medir cómo manejan los accesos a la base 
de datos en igualdad de condiciones.\\
\newpage

%Evaluación
%	Preparación de ambiente de desarrollo
%	Posibles errores y como solucionarlos
\section{Evaluación}

\newpage

%Pruebas
%	Definición de tipo de pruebas
%	Definir cómo se harán las pruebas
\section{Pruebas}

\newpage

%Resultados
\section{Resultados}

\newpage

%Conclusión	
\section{Conclusiones}

\newpage

\section{Bibliografía}


[1]http://www.imh.es/elearning-es/que-es-elearning hora de consulta 03:25 17/10/2012

[2]Propuestas metodológicas para el desarrollo de aplicaciones Web: una evaluación según la ingeniería de métodos
   M. Mendoza* y J. Barrios**
   Postgrado en Computación,
   Universidad de Los Andes, Facultad de Ingeniería,
   Escuela de Ingeniería de Sistemas,
   Mérida, Venezuela.
   *nella\_1@hotmail.com,
   **ijudith@ing.ula.ve
   Revista Ciencia e Ingeniería. Vol. 25 No. 2. 2004 

[3]http://www.koala-soft.com/de-web-10-a-web-30 hora de consulta 17:00 17/10/2012

[4] http://www.quantium.com.mx/2012/08/06/nodejs/ (hora de consulta 18:18 28/11/2012)

[5]La evolución de los servicios de referencia digitales en la Web 2.0
   Catuxa Seoane García
   catuxa@gmail.com
   Documentalista. Bibliotecas Municipales de A Coruña
   Vanesa Barrero Robledo
   uveybe@gmail.com
   Documentalista. Yahoo España
   Autoras del blog Deakialli DocuMental (http://www.deakialli.com)

[6]http://www.oreillynet.com/oreilly/tim/news/2005/09/30/what-is-web-20.html  hora de consulta 18:10 17/10/2012

[7]http://www.ecured.cu/index.php/Qooxdoo hora de consulta 02:45 22/10/2012

[8]http://www.desarrolloweb.com/articulos/que-es-html.html hora de consulta 22:37 17/10/2012

[9]http://www.monografias.com/trabajos7/html/html.shtml hora de consulta 22:44 17/10/2012

[10] https://npmjs.org (hora de consulta 18:10 28/11/2012)

[11]http://www.monografias.com/trabajos62/sistemas-informacion-web/sistemas-informacion-web2.shtml hora de consulta 01:59 18/10/2012

[12]Procesos Ágiles Para el Desarrollo de Aplicaciones Web 
    Paloma Cáceres, Esperanza Marcos
    Grupo Kybele
    Departamento de Ciencias Experimentales e Ingeniería
    Universidad Rey Juan Carlos
    C/ Tulipán, s/n, 28933 – Móstoles, Madrid (España)
    {p.caceres/cuca}@escet.urjc.es

[13]http://www.genbetadev.com/bases-de-datos/una-introduccion-a-mongodb hora de consulta 03:29 22/10/2012

[14]I Congreso Internacional de Ciberperiodismo y Web 2.0. Bilbao: Noviembre 2009
    ¿Web 2.0, Web 3.0 o Web Semántica?: El impacto en los sistemas de
    información de la Web
    Por Lluís Codina
    Universidad Pompeu Fabra
    www.lluiscodina.com | www.lluiscodina.com/diagramas/htm
    Página 4
    
[15]http://www.kabytes.com/programacion/kendo-ui-framework-html5-css3-y-jquery/ hora de consulta 03:35 22/10/2012

[16]http://interfacemindbraincomputer.wetpaint.com/page/2.A.1.1.7.-Computacion+Ubicua+o+Pervasiva+e+Inteligencia+ambiental hora de consulta 
21:22 24/10/2012

[17] http://nodejs.org/download/ (hora de consulta 19:10 29/11/2012) 


%Anexos
%	Instalaciones, Hola Mundo, problemas y soluciones  de herramiemntas que no se vieron a fondo (informe)



\end{document}
