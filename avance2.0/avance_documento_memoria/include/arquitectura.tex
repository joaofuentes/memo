\section{Arquitectura}

%explicar en parrafo pequeño que es una arquitectura de software
%\textit{agregar parrafo introductorio}\\
Una arquitectura de software corresponde a un patrón de referencia que brinda
un marco necesario para guiar la construcción de software y establece la estrucutura de 
funcionamiento  e interacción entre sus diversas partes [20]. Este concepto 
"se refiere a la estructuración del sistema que, idealamente, se crea en las etapas 
tempranas de desarrollo"[21].\\ 

Al crear un software (independientemente de la metodología que se utilice), es necesario
cumplir una serie de pasos  que preceden a su construcción:
\begin{itemize}
 \item Requerimientos: Se enfoca a la captura y priorización de necesidades a satisfacer,
  ya sean de calidad, rendimiento y/o reestrictivas. Estos requerimientos son preponderantes
  e influencia la desición acerca de que arquitectura a utilizar.
 \item Diseño: Corresponde a la fase central en relación con la arquitectura. Debe satisfacer todos
  los requerimientos y no solo utilizar tecnologías de moda.
 \item Documentación: La documentación un factor crucial para comunicar un diseño de forma exitosa.
  Generalemente se utiliza la representación de varias de sus estructuras mediante el uso vistas. 
  Una vista generalmente contiene un diagrama, además de información adicional, que apoya en la 
  comprensión de dicho diagrama.
 \item Evaluación: Es de suma importancia evaluar el diseño una vez que este ha sido documentado 
  con el fin de identificar posibles problemas y riesgos. Evaluar (y validar) el diseño antes
  de codificar, disminuye el costo de corrección de errores.
\end{itemize}


Los grandes objetivos de una arquitectura de software son :
\begin{enumerate}
 \item Servir como guía durante el proceso de desarrollo. 
 \item Definir y satisfacer los atributos de calidad. 
\end{enumerate}
  

%explicar pq es necesario definir una arquitectura
Como dice Danny Thorpe:
\begin{center}
 \textit{"Programar sin una arquitectura en mente, es como explorar una gruta sólo con una
 linterna: no sabes dónde estás, dónde has estado ni hacia dónde vas"} [18]
\end{center}
La importancia de contar con una buena arquitectura que se adecue a situaciones reales, donde 
las diversas herramientas serán probadas. Es por ello que la arquitectura debe ser la respuesta 
ante un problema, no una imposición. Por otra parte, el desarrollo de software dejó de ser, hace mucho
tiempo, el trabajo de una o dos personas, pasando a ser equipos; por tanto es necesario facilitar la
comunicación entre sus integrantes.\\

A la hora de diseñar un software, hay que tener en cuenta que la arquitectura a utilizar se 
describe utilizando varios tipos de modelos[22]:
\begin{itemize}
 \item Estructurales: Se centran en la estructura coherente del sistema completo, en lugar de centrarse 
  en su composición. Representan todo como una colección organizada de componentes. 
 \item Frameworks: Identifican patrones de diseño repetibles, los cuales se pueden encontrar en aplicaciones
  similares.
 \item Dinámicos: Se centran en los aspectos del comportamiento dinámico de la arquitectura,  indicando
  como la estructura o la configuración del sistema puede cambiar en función de eventos externos.
 \item De Procesos: Se enfocan en el diseño de los procesos del negocio que el sistema debe soportar.
\end{itemize}

Dependiendo de la situación, se debe tomar la desición de que tipo de arquitectura utilizar.\\



Algunos elementos de una arquitectura son:
\begin{itemize}
 \item Cliente y Servidor
 \item Base de Datos
 \item Componentes
 \item Sistemas de Nivel Jerárquico
\end{itemize}

De entre las diversas arquitecturas, algunas de las más relevantes son:
\begin{itemize}
 \item Orientada a Servicios
 \item Modelo Vista Controlador
 \item Cliente-Servidor
 \item Modelo de N-Capas
\end{itemize}
%\textit{Pregntar al profe si es necesario explicar cada uno} basta con explicar y adjuntar diagrama

\subsection{Arquitectura Orientada a Servicios}

Es una arquitectura de software que permite la creación y/o cambios de los procesos de negocio desde la perspectiva 
de tecnologías de la información de forma ágil, a través de la composición de nuevos procesos utilizando las funcionalidades 
de negocio que están contenidas en la infraestructura de aplicaciones actuales, utilizando protocolos estándar e interfaces 
convencionales \footnote{usualmente Web Services} para facilitar el acceso a la lógica de negocios y la información entre diversos 
servicios[27].\\

Por lo general, en una empresa coexisten un sin número de aplicaciones, lo que conlleva a una serie de inconvenientes que aumentan el 
el tiempo y esfuerzo en que se responde a un requerimiento en particular.\\

Uno de los principales inconvenientes, es que, ante aplicaciones desarrolladas en lenguajes diferentes, no se pueda acceder desde una
a otra para consultar un dato en particular[24].\\

\textit{agregar imagen}\\

Mediante la aplicación de la Arquitectura Orientada a Servicios (SOA)\footnote{Service Oriented Architecture, por sus siglas en inglés} 
pretende solucionar los problemas antes mencionados. Dentro de la arquitectura SOA la funcionalidad se implementa en pequeños componentes 
autónomos reutilizables denominados servicios.\\

SOA no es software o un lenguaje de programación, sino un marco de trabajo conceptual que permite a organizaciones unir los
objetivos de negocio con su infraestructura de tecnologías de información, integrando los datos y la lógica de negocio de sus 
sistemas separados.[25]\\

La necesidad de tal marco se deriva de la evolución del software de negocio. Antes, los desarrollos de aplicaciones de negocio se 
concentraban en necesidades específicas: contabilidad, compras, planillas de sueldos. Cada aplicación se desarrollaba sin considerar a 
otros sistemas dentro de la empresa, pues las aplicaciones (de pequeña escala), se caracterizaban por ser auto suficientes; el cambio 
más grande es filosófico, ya que, en lugar de pensar en el diseño de aplicaciones individuales para resolver problemas especificos, 
SOA ve el software como un patrón que soporta todo el proceso del negocio. Cada elemento de un servicio es un componente que puede 
ser utilizado muchas veces a través de muchas funciones y procesos dentro y fuera de la empresa\\

La idea detrás de todo esto es que es más efectivo trabajar con servicios que con aplicaciones.\\

SOA define las siguientes capas de software [26]:
%revisar y cruzar la info con mas fuentes, ej [27]
\begin{itemize}
 \item Aplicaciones básicas, sistemas desarrollados bajo cualquier arquitectura o tecnología, geográficamente dispersos y bajo 
  cualquier figura de propiedad.
 \item Exposición de funcionalidades, donde las funcionalidades de la capa aplicativas son expuestas en forma de servicios (webservices).
 %REVISAR LA 2DA
 \item Integración de servicios que facilita el intercambio de datos entre elementos de la capa aplicativa orientada a procesos 
  empresariales internos o en colaboración;
 \item Composición de procesos, que define el proceso en términos del negocio y sus necesidades, y que varia en función del negocio;
 \item Entrega, donde los servicios son desplegados a los usuarios finales.
\end{itemize}

\subsubsection{Principios  SOA}

Algunos principios importantes de esta arquitectura son [28][29]:
%RE REDACTAR ESTA PARTE
\begin{itemize}
 \item \textbf{Deben ser reusables}: Los servicios deben ser diseñados y construidos pensando en su reutilización dentro de la misma 
    aplicación, dentro del dominio de aplicaciones de la empresa o incluso dentro del dominio público para su uso masivo.
    
 \item \textbf{Deben proporcionar un contrato formal}: Los servicios desarrollados deben proporcionar un contrato en el cual figuren: 
    el nombre del servicio, su forma de acceso, las funcionales que ofrece, los datos de entrada de cada una de las funcionalidades y 
    los datos de salida. Así, los consumidores del servicio, accederán a través de las reglas establecidas por el contrato, logrando 
    la independencia entre el consumidor y la implementación del propio servicio.
    
 \item \textbf{Deben tener bajo acoplamiento}: Los servicios tienen que ser independientes  unos de otros. Para lograrlo, es necesario que,
    cada vez que se vaya a ejecutar un servicio, se accederá a él a través del contrato, logrando la independencia entre el servicio que se 
    va a ejecutar y el que lo llama. De esta manera serán totalmente reutilizables.
    
 \item \textbf{Deben permitir la composición}: Los servicios deben ser construidos de tal manera que puedan ser utilizados para construir 
    servicios genéricos de más alto nivel, el cual estará compuesto de servicios de más bajo nivel. Al trabajar con Servicios Web,  
    se logra mediante el uso de los protocolos para orquestación \footnote{``En servicios Web, la orquestación se refiere a un conjunto de
    actividades que conforman un proceso. Orquestación es la gestión, la organización de las interacciones que existen entre los servicios 
    web de un proceso.''[30] } y coreografía \footnote{''La coreografía corresponde a la descripción y guía de un modelo global y define las
    colaboraciones entre cualquier tipo de aplicaciones. Es un protocolo de negocio que dicta las reglas de interacción que deben ser
    cumplidas por las entidades participantes.``[30]} .

 \item textbf{Deben de ser autónomos}: Los servicios deben tener su propio entorno de ejecución. De esta manera el servicio es totalmente 
    independiente y es posible asegurar que  podrá ser reutilizable desde el punto de vista de la plataforma de ejecución.

 \item \textbf{No deben tener estado}: Los servicios no deben guardar ningún tipo de información. La razón de esto corresponde a que si
    una aplicación está formada por un conjunto de servicios, y un servicio almacena algún tipo de información, es posible que se produzcan 
    problemas de inconsistencia de datos. Es por ello que es necesario que servicio sólo contenga lógica, y que toda información esté 
    almacenada en algún sistema de información sea del tipo que sea.

 \item \textbf{Deben poder ser descubiertos}: Los servicios deben poder ser descubiertos de alguna forma para que puedan ser utilizados, 
    para evitar la creación de multiples servicios que proporcionen las mismas funcionalidades. 
    %En el caso de los Servicios Web, el descubrimiento se logrará publicando los interfaces de los servicios en registros UDDI.
\end{itemize}



\subsection{Arquitectura Modelo Vista Controlador (MVC)}

Es un patrón de arquitectura de software que separa la lógica de negocio de la interfaz de usuario, facilitando la 
evolución por separado de ambos aspectos e incrementa reutilización y flexibilidad [31].\\

Su principal característica es que separa los datos de la aplicación, la interfaz de usuario, y la lógica de control en tres componentes 
distintos. Es encontrado frecuentemente en aplicaciones web, donde la vista es la página HTML (que vé el usuario) y el código que provee 
de datos dinámicos a la página; el modelo que corresponde  al sistema de gestión de base de datos y lógica de negocio; y el controlador 
que es el responsable de recibir los eventos de entrada desde la vista.\\

MVC fue introducido inicialmente en la comunidad de desarrolladores de Smalltalk-80 [32].\\

%agregar más info

\subsubsection{Componentes del MVC}

\textit{agregar imagen MVC}

Si bien, hay una gran cantidad de variaciones al modelo comúnmente conocido del patrón arqitectónico MVC,\\
%sólo el interior de cada comoponente se vuelve más complejo. MEJORAR!!!

La finalidad del modelo es mejorar la reusabilidad por medio del desacople entre la vista y el modelo. Los elementos de esta
arquitectura son [33]:

\begin{enumerate}
 \item \textbf{El Modelo}: Encapsula los datos y las funcionalidades. Es responsable de:
	\begin{itemize}
	 \item Acceder a la capa de almacenamiento de datos.
	 \item Definir las reglas de negocio, es decir, la funcionalidad del sistema.
	 \item Llevar un registro, tanto de las vistas como de los controladores del sistema.
	 %Agregar más de acuerdo a la busqueda bibliográfica
	\end{itemize}
       El modelo es independiente de cualquier representación de salida y/o comportamiento de entrada
 
 \item \textbf{La Vista}: Muestra la información al usuario y pueden existir múltiples vistas del modelo. Es responsable de:
	 \begin{itemize}
	  \item Recibir los datos del modelo y mostrarlos al usuario.
	  \item Tener un registro de su controlador asociado.
	  \item Suministrar un servicio de “Actualización()”, para que sea invocado por el controlador o por el modelo.
	 \end{itemize}
       Cada vista tiene asociado un componente  controlador.
       
 \item \textbf{El Controlador}: Recibe las entradas, usualmente como eventos codificados. Es responsable de:
	   \begin{itemize}
	    \item Recibir los eventos de entrada, que por lo general son un click, un cambio dentro de un campo de texto, o el
	          movimiento del ratón (entre otros).
	    \item Contener reglas de gestión de eventos del tipo ''IF``, por ejemplo: ''IF evento X, THEN acción Y`` 
	   \end{itemize}
       Los eventos son traducidos a solicitudes de servicio ("service requests") para el modelo o la vista.
\end{enumerate}

\subsubsection{Flujo del MVC}

El flujo que sigue el control generalmente es el siguiente [33], el cual puede cambiar dependiendo de la implementación del:

\begin{enumerate}
 \item El \textbf{usuario} interactúa con la interfaz de usuario, es decir la \textbf{vista}, a través de un evento: por ejemplo, 
       el usuario hace click en un enlace).
 
 \item El \textbf{controlador} recibe  a través de la \textbf{vista} la notificación de la acción solicitada por el \textbf{usuario}. 
 
 \item El \textbf{controlador} gestiona el evento que llega a través de un gestor de eventos o callback (generalmente).
 
 \item El \textbf{controlador} accede al \textbf{modelo}, y lo actualiza de forma adecuada a la acción solicitada por el \textbf{usuario}. 
       Por ejemplo, el \textbf{controlador} actualiza el carrito de compras del \textbf{usuario}). 
       
 \item El \textbf{controlador} delega a la \textbf{vista} la tarea de desplegar la interfaz usuaria. La \textbf{vista} obtiene sus datos 
      del \textbf{modelo} para generar la interfaz apropiada para el \textbf{usuario} donde se refleja los cambios en el \textbf{modelo}.
      Por ejemplo, producir un listado del contenido del carrito de compras.
 
 \item La interfaz usuaria (\textbf{vista}) espera nuevas interacciones del usuario, comenzando el ciclo nuevamente

\end{enumerate}



\subsection{Arquitectura Cliente-Servidor}

%Agregar importancia de carga balanceada entre ambos
IBM define al modelo Cliente-Servidor como[34][35][36]: 

\begin{center}
      \textit{''Es la tecnología que proporciona al usuario final el acceso transparente a las 
     aplicaciones, datos, servicios de cómputo o cualquier otro recurso del grupo de trabajo y/o, a través de la organización, 
     en múltiples plataformas. El modelo soporta un medio ambiente distribuido en el cual los requerimientos de servicio hechos 
     por estaciones de trabajo inteligentes o "clientes'', resultan en un trabajo realizado por otros computadores llamados 
     servidores``}.
\end{center}

Es una arquitectura distribuida que permite a los usuarios finales obtener acceso a la información de forma transparente aún en 
entornos multiplataforma. Se trata de la arquitectura más extendida en la implementación de Sistemas Distribuidos[37].\\

\textit{agregar una imagen de la arquitectura y del balanceo}\\

%agregar info de funcionamiento en general
Es un modelo para el desarrollo de sistemas de información en el que las operaiones se dividen en procesos independientes y cooperativos
entre sí para intercambiar ya sea información, servicios o recursos. Se denomina cliente al proceso que inicia la ''conversación`` solicita 
los recursos; y servidor al proceso que responde a dichas solicitudes [36].\\ 

''En este modelo las aplicaciones se dividen de forma que el servidor contiene la parte que debe ser compartida por varios usuarios, y en el 
cliente permanece sólo lo particular de cada usuario``\footnote{http://bit.ly/19mazo5 (hora de consulta 19:26 24/06/2013)} .\\


\subsubsection{Cliente}

Es el que inicia un requerimiento de servicio, formulando los requerimientos y pasandolos al servidor;  se lo conoce con el término 
front-end. [34]. La ubicación de los datos o de las aplicaciones es totalmente transparente para el cliente.\\

El cliente maneja, por lo general, todas las funciones que tienen que ver con la manipulación y visualización de datos, por ende están 
desarrolladas sobre plataformas que permiten construir interfaces gráficas de usuario (GUI)\footnote{Graphical User Interface, por sus
siglas en inglés.}\\

Dentro de las tareas realizadas por el cliente, destacan[37]:
\begin{itemize}
 \item Administrar la GUI.
 \item Interactuar con el usuario.
 \item Procesar la lógica de la aplicación y, en algunos casos, hacer validaciones locales.
 \item Generar requerimientos de bases de datos.
 \item Recibir resultados del servidor y mostrarlos al cliente.
\end{itemize}


%Dependiendo del contexto de la aplicación y la funcionalidad del cliente, éste puede ser clasificado en:
%investigar mas!
%\begin{itemize}
% \item \textbf{Basado en aplicación de usuario}: En caso de que los datos sean de baja interacción y esten fuertemente relacionados 
%    con la actividad de los usuarios de esos clientes. 
% \item \textbf{Basado en lógica de negocio}:  Toma datos suministrados por el usuario y/o la base de datos y efectúa los cálculos 
%    necesarios según los requerimientos del usuario.
%\end{itemize}


\subsubsection{Servidor}

Es una aplicación que ofrece un servicio a usuarios de Internet, el servidor es un programa que recibe una solicitud, realiza el 
servicio requerido y devuelve los resultados en forma de una respuesta. Generalmente un servidor puede tratar múltiples peticiones 
(múltiples clientes) al mismo tiempo.\\


\subsubsection{Principios Cliente-Servidor}

Algunas características de este tipo e arquitectura son[36][37]:
\begin{itemize}
 \item El servidor proporciona los servicios y el cliente los utiliza.
 \item La única relación entre clientes y servidores es la que se establece a través del intercambio de mensajes. El mensaje es el 
       mecanismo para la petición y entrega de solicitudes de servicio. 
 \item El uso de recursos compartidos. Muchos clientes pueden usar los mismos servidores y, a través de ellos, comparten tanto recursos 
       lógicos como físicos. Estos recursos son regulados por el servidor.
 \item El cliente inicia las comunicaciones, el servidor espera de forma pasiva.
 \item Los detalles de la implementación son invisibles para el cliente, es decir que existe encapsulamiento de servicios.
 \item Transparencia de física y de localización, es decir que el cliente no sabe si el servidor está distribuido y/o su localización 
       física.
 \item Los programas y datos centralizados en servidores, facilitan tanto la integridad, como su mantenimiento.
 \item Ambiente heterogéneo, tanto la plataforma de hardware, como el sistema operativo del cliente y del servidor no son necesariamente 
       la misma. Esto representa una de las principales ventajas de esta arquitectura, permitiendo conectar clientes y servidores 
       independientemente de sus plataformas. 
\end{itemize}

\subsubsection{Flujo de cliente servidor}

Si bien pueden haber ciertas diferencas dependiendo de la implementación, el flujo  de funcionamiento de un sistema 
Cliente/Servidor es [37]:

\begin{enumerate}
 \item El \textbf{cliente} solicita  información al \textbf{servidor}.
 \item El \textbf{servidor} recibe la petición del \textbf{cliente}.
 \item El \textbf{servidor} procesa dicha solicitud.
 \item El \textbf{servidor} envía el resultado obtenido al \textbf{cliente}.
 \item El \textbf{cliente} recibe el resultado y lo procesa.
\end{enumerate}

   
\subsection{Arquitectura N-Capas}

%Explicar la diferencia entre capa y nivel
%


\subsubsection{Elección}

Considerando el hecho de que uno de los objetivos de esta memoria es, realizar pruebas y obtener
resultados concretos acerca de diversas herramientas de desarrollo web, es necesario enmarcar
dichas pruebas bajo un estándar que asegure una base pareja para todas ellas.\\

Es por dicha razón que se utilizará una arquitectura basada en 3 capas, de tipo \textbf{estructural} y 
\textbf{framework}.\\

\subsection{Arquitectura de 3 capas y 2 niveles}
Una arquitectura bastante popular y de carácter relativamente simple, es la de 3 capas, cuyo
objetivo primordial es la separación de la aplicación en [23]:

\textit{Crear y agregar una imagen (con su respectiva explicación)}\\

\begin{enumerate}
 \item Una capa de presentación: Corresponde a la capa que ve el usuario, es decir, presenta 
  el sistema al usuario. Se responsabiliza de que se le comunique información al usuario por 
  parte del sistema y viceversa. La tarea de capturar información sobre el usuario se realiza,
  por lo general a través de formularios. Esta capa se comunica exclusiamente con la capa lógica.
  
 \item Una capa lógica/de negocios: Es la encargada de recibir, procesar y responder las peticiones
  que recibe, tanto del usuario a través de la capa de presentación, como las de la Base de Datos mediante
  la capa de datos. En otras palabras es una capa intermedia.
 
 \item Una capa de datos: Es la encargada acceder a los datos y está formada por un (o más) gestor de 
  base de datos. Dicho gestor recibe las peticiones de almacenamiento, recuperación, actualización y/o 
  eliminación de datos; desde la capa lógica. 
\end{enumerate}

Basicamente el objetivo principal es separar las distintas lógicas de la aplicación en niveles y que
éstos posean estructuras bien planteadas.\\

Para el caso de esta memoria, las pruebas se centrarán en la capa lógica y de presentación. Dejando
la capa de datos como una constante.
%Uso de arquitectura 3-tier
  %explicar de que se trata y pq se usa
  %hacer el diagrama 

%mapear las capas a la aplicaciòn



\newpage