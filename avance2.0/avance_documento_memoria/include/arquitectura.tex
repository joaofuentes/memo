\section{Arquitectura}

%explicar en parrafo pequeño que es una arquitectura de software
%\textit{agregar parrafo introductorio}\\
Una arquitectura de software corresponde a un patrón de referencia que brinda
un marco necesario para guiar la construcción de software y establece la estrucutura de 
funcionamiento  e interacción entre sus diversas partes [20]. Este concepto 
"se refiere a la estructuración del sistema que, idealamente, se crea en las etapas 
tempranas de desarrollo"[21].\\ 

Al crear un software (independientemente de la metodología que se utilice), es necesario
cumplir una serie de pasos  que preceden a su construcción:
\begin{itemize}
 \item Requerimientos: Se enfoca a la captura y priorización de necesidades a satisfacer,
  ya sean de calidad, rendimiento y/o reestrictivas. Estos requerimientos son preponderantes
  e influencia la desición acerca de que arquitectura a utilizar.
 \item Diseño: Corresponde a la fase central en relación con la arquitectura. Debe satisfacer todos
  los requerimientos y no solo utilizar tecnologías de moda.
 \item Documentación: La documentación un factor crucial para comunicar un diseño de forma exitosa.
  Generalemente se utiliza la representación de varias de sus estructuras mediante el uso vistas. 
  Una vista generalmente contiene un diagrama, además de información adicional, que apoya en la 
  comprensión de dicho diagrama.
 \item Evaluación: Es de suma importancia evaluar el diseño una vez que este ha sido documentado 
  con el fin de identificar posibles problemas y riesgos. Evaluar (y validar) el diseño antes
  de codificar, disminuye el costo de corrección de errores.
\end{itemize}


Los grandes objetivos de una arquitectura de software son :
\begin{enumerate}
 \item Servir como guía durante el proceso de desarrollo. 
 \item Definir y satisfacer los atributos de calidad. 
\end{enumerate}
  

%explicar pq es necesario definir una arquitectura
Como dice Danny Thorpe:
\begin{center}
 \textit{"Programar sin una arquitectura en mente, es como explorar una gruta sólo con una
 linterna: no sabes dónde estás, dónde has estado ni hacia dónde vas"} [18]
\end{center}
La importancia de contar con una buena arquitectura que se adecue a situaciones reales, donde 
las diversas herramientas serán probadas. Es por ello que la arquitectura debe ser la respuesta 
ante un problema, no una imposición. Por otra parte, el desarrollo de software dejó de ser, hace mucho
tiempo, el trabajo de una o dos personas, pasando a ser equipos; por tanto es necesario facilitar la
comunicación entre sus integrantes.\\

A la hora de diseñar un software, hay que tener en cuenta que la arquitectura a utilizar se 
describe utilizando varios tipos de modelos[22]:
\begin{itemize}
 \item Estructurales: Se centran en la estructura coherente del sistema completo, en lugar de centrarse 
  en su composición. Representan todo como una colección organizada de componentes. 
 \item Frameworks: Identifican patrones de diseño repetibles, los cuales se pueden encontrar en aplicaciones
  similares.
 \item Dinámicos: Se centran en los aspectos del comportamiento dinámico de la arquitectura,  indicando
  como la estructura o la configuración del sistema puede cambiar en función de eventos externos.
 \item De Procesos: Se enfocan en el diseño de los procesos del negocio que el sistema debe soportar.
\end{itemize}

Dependiendo de la situación, se debe tomar la desición de que tipo de arquitectura utilizar.\\



Algunos elementos de una arquitectura son:
\begin{itemize}
 \item Cliente y Servidor
 \item Base de Datos
 \item Componentes
 \item Sistemas de Nivel Jerárquico
\end{itemize}

De entre las diversas arquitecturas, algunas de las más relevantes son:
\begin{itemize}
 \item Orientada a Servicios
 \item Modelo Vista Controlador
 \item Cliente-Servidor
 \item Modelo de N-Capas
\end{itemize}
%\textit{Pregntar al profe si es necesario explicar cada uno} basta con explicar y adjuntar diagrama

\subsection{Arquitectura Orientada a Servicios}

Es una arquitectura de software que permite la creación y/o cambios de los procesos de negocio desde la perspectiva 
de tecnologías de la información de forma ágil, a través de la composición de nuevos procesos utilizando las funcionalidades 
de negocio que están contenidas en la infraestructura de aplicaciones actuales, utilizando protocolos estándar e interfaces 
convencionales \footnote{usualmente Web Services} para facilitar el acceso a la lógica de negocios y la información entre diversos 
servicios[27].\\

Por lo general, en una empresa coexisten un sin número de aplicaciones, lo que conlleva a una serie de inconvenientes que aumentan el 
el tiempo y esfuerzo en que se responde a un requerimiento en particular.\\

Uno de los principales inconvenientes, es que, ante aplicaciones desarrolladas en lenguajes diferentes, no se pueda acceder desde una
a otra para consultar un dato en particular[24].\\

\textit{agregar imagen}\\

Mediante la aplicación de la Arquitectura Orientada a Servicios (SOA)\footnote{Service Oriented Architecture, por sus siglas en inglés} 
pretende solucionar los problemas antes mencionados. Dentro de la arquitectura SOA la funcionalidad se implementa en pequeños componentes 
autónomos reutilizables denominados servicios.\\

SOA no es software o un lenguaje de programación, sino un marco de trabajo conceptual que permite a organizaciones unir los
objetivos de negocio con su infraestructura de tecnologías de información, integrando los datos y la lógica de negocio de sus 
sistemas separados.[25]\\

La necesidad de tal marco se deriva de la evolución del software de negocio. Antes, los desarrollos de aplicaciones de negocio se 
concentraban en necesidades específicas: contabilidad, compras, planillas de sueldos. Cada aplicación se desarrollaba sin considerar a 
otros sistemas dentro de la empresa, pues las aplicaciones (de pequeña escala), se caracterizaban por ser auto suficientes; el cambio 
más grande es filosófico, ya que, en lugar de pensar en el diseño de aplicaciones individuales para resolver problemas especificos, 
SOA ve el software como un patrón que soporta todo el proceso del negocio. Cada elemento de un servicio es un componente que puede 
ser utilizado muchas veces a través de muchas funciones y procesos dentro y fuera de la empresa\\

La idea detrás de todo esto es que es más efectivo trabajar con servicios que con aplicaciones.\\

SOA define las siguientes capas de software [26]:
%revisar y cruzar la info con mas fuentes, ej [27]
\begin{itemize}
 \item Aplicaciones básicas, sistemas desarrollados bajo cualquier arquitectura o tecnología, geográficamente dispersos y bajo 
  cualquier figura de propiedad.
 \item Exposición de funcionalidades, donde las funcionalidades de la capa aplicativas son expuestas en forma de servicios (webservices).
 %REVISAR LA 2DA
 \item Integración de servicios que facilita el intercambio de datos entre elementos de la capa aplicativa orientada a procesos 
  empresariales internos o en colaboración;
 \item Composición de procesos, que define el proceso en términos del negocio y sus necesidades, y que varia en función del negocio;
 \item Entrega, donde los servicios son desplegados a los usuarios finales.
\end{itemize}

\subsubsection{Principios  SOA}

Algunos principios de esta arquitectura son [28][29]:
%RE REDACTAR ESTA PARTE
\begin{itemize}
 \item Los Servicios deben ser reusables: Todo servicio debe ser diseñado y construido pensando en su reutilización dentro de la misma 
  aplicación, dentro del dominio de aplicaciones de la empresa o incluso dentro del dominio público para su uso masivo.
 \item Los Servicios deben proporcionar un contrato formal: Todo servicio desarrollado, debe proporcionar un contrato en el cual figuren: 
    el nombre del servicio, su forma de acceso, las funcionales que ofrece, los datos de entrada de cada una de las funcionalidades y 
    los datos de salida. De esta manera, todo consumidor del servicio, accederá a este mediante el contrato, logrando así la independencia 
    entre el consumidor y la implementación del propio servicio.
   
   \item Los Servicios deben tener bajo acoplamiento: Es decir, que los servicios tienen que ser independientes los unos de los otros. Para 
    lograr ese bajo acoplamiento, lo que se hará es que cada vez que se vaya a ejecutar un servicio, se accederá a él a través del contrato, 
    logrando así la independencia entre el servicio que se va a ejecutar y el que lo llama. De esta manera serán totalmente reutilizables.

   \item Los Servicios deben permitir la composición: Todo servicio debe ser construido de tal manera que pueda ser utilizado para construir 
    servicios genéricos de más alto nivel, el cual estará compuesto de servicios de más bajo nivel. En el caso de los Servicios Web, esto 
    se logrará mediante el uso de los protocolos para orquestación(WS-BPEL) y coreografía (WS-CDL).

   \item Los Servicios deben de ser autónomos: Todo Servicio debe tener su propio entorno de ejecución. De esta manera el servicio es totalmente 
    independiente y nos podemos asegurar que así podrá ser reutilizable desde el punto de vista de la plataforma de ejecución.

    \item Los Servicios no deben tener estado: Un servicio no debe guardar ningún tipo de información. Esto es así porque una aplicación está 
    formada por un conjunto de servicios, lo que implica que si un servicio almacena algún tipo de información, se pueden producir 
    problemas de inconsistencia de datos. La solución, es que un servicio sólo contenga lógica, y que toda información esté almacenada 
    en algún sistema de información sea del tipo que sea.

    \item Los Servicios deben poder ser descubiertos: Todo servicio debe poder ser descubierto de alguna forma para que pueda ser utilizado, 
    consiguiendo así evitar la creación accidental de servicios que proporcionen las mismas funcionalidades. En el caso de los Servicios 
    Web, el descubrimiento se logrará publicando los interfaces de los servicios en registros UDDI.
\end{itemize}





\subsection{Arquitectura Modelo Vista Controlador}

\subsection{Arquitectura Cliente-Servidor}

\subsection{Arquitectura N-Capas}

\subsubsection{Arquitectura 3-Capas}

Considerando el hecho de que uno de los objetivos de esta memoria es, realizar pruebas y obtener
resultados concretos acerca de diversas herramientas de desarrollo web, es necesario enmarcar
dichas pruebas bajo un estándar que asegure una base pareja para todas ellas.\\

Es por dicha razón que se utilizará una arquitectura basada en 3 capas, de tipo \textbf{estructural} y 
\textbf{framework}.\\

\subsection{Arquitectura de 3 capas}
Una arquitectura bastante popular y de carácter relativamente simple, es la de 3 capas, cuyo
objetivo primordial es la separación de la aplicación en [23]:

\textit{Crear y agregar una imagen (con su respectiva explicación)}\\

\begin{enumerate}
 \item Una capa de presentación: Corresponde a la capa que ve el usuario, es decir, presenta 
  el sistema al usuario. Se responsabiliza de que se le comunique información al usuario por 
  parte del sistema y viceversa. La tarea de capturar información sobre el usuario se realiza,
  por lo general a través de formularios. Esta capa se comunica exclusiamente con la capa lógica.
  
 \item Una capa lógica/de negocios: Es la encargada de recibir, procesar y responder las peticiones
  que recibe, tanto del usuario a través de la capa de presentación, como las de la Base de Datos mediante
  la capa de datos. En otras palabras es una capa intermedia.
 
 \item Una capa de datos: Es la encargada acceder a los datos y está formada por un (o más) gestor de 
  base de datos. Dicho gestor recibe las peticiones de almacenamiento, recuperación, actualización y/o 
  eliminación de datos; desde la capa lógica. 
\end{enumerate}

Basicamente el objetivo principal es separar las distintas lógicas de la aplicación en niveles y que
éstos posean estructuras bien planteadas.\\

Para el caso de esta memoria, las pruebas se centrarán en la capa lógica y de presentación. Dejando
la capa de datos como una constante.
%Uso de arquitectura 3-tier
  %explicar de que se trata y pq se usa
  %hacer el diagrama 

%mapear las capas a la aplicaciòn



\newpage